%!TEX root = ./main.tex

\section{Related Work}
\label{sec:related}

\btext{Bologna? Related work di deviant accorciati + i danesi + deviant?}



When process models are loosely-structured, procedural discovery could produce spaghetti-models~\cite{2012-Maggi,deviant-tkde}. In that case, declarative approaches are more suitable for the purpose because they allow to briefly list all the required or prohibited behaviours in the business process. 

Over the last decade, several works focused on declarative process discovery~\cite{2011-Maggi,2012-Maggi,2012-Schunselaar,2017-DiCiccio,2015-DiCiccio}.
In~\cite{2011-Maggi}, Maggi et al. propose to build a set of all possible candidate Declare constraints considering all the activities that appear in the log, translate the constraints in \ac{LTL}, and check them against the whole log until certain levels of recall and specificity are reached. 
Techniques to refine the business model excluding vacuously satisfied constraints are the focus of the subsequent works by Schunselaar et al. \cite{2012-Schunselaar} and Maggi et al. \cite{2012-Maggi}, whereas Di Ciccio et al. \cite{2017-DiCiccio} propose an approach to filter out frequent redundancies and inconsistencies. 
All the cited declarative approaches do not deal with negative examples. Nonetheless, interestingly from our point of view, in \cite{2015-DiCiccio}  the authors define among the metrics to guide the declarative discovery approach, besides support and confidence, also the interest factor for each constraint w.r.t. the log, and the possibility to include in the search space constraints on prohibited behaviours.

Negative examples are instead actively employed in the declarative discovery approaches \cite{2007-Lamma,2009-Chesani,2010-Bellodi,2016-Bellodi}, %, which are all based on the functioning principles of \ac{ICL} algorithm \cite{1995-DaRaedt} and thus intrinsically dependent to the availability of both negative and positive examples. 
eventually relaying for the evaluation on the output of synthetical log generators able to produce both positive and negative process cases \cite{2009-Goedertier,2014-Stocker,2010-Hee,2019-Chesani,2017-Chesani,2020-Loreti}.
The technique by Lamma et al. \cite{2007-Lamma,2007b-Lamma} learns integrity constraints expressed as logical formulas, and translates them into the equivalent DecSerFlow constructs \cite{2006-Aalst}. 
Bellodi et al. \cite{2010-Bellodi}, \cite{2016-Bellodi} employ the same approach and automatically convert the results into Markov Logic formulas---statistical relational learning is used to determine the weight of each formula.
Analogously, Chesani et al. \cite{2009-Chesani} propose to learn a set of SCIFF rules \cite{2008-Alberti} and translate them into ConDec constraints \cite{2006-Pesic}. The approach that we adopt in this work instead, is the one presented in \cite{deviant-tkde}, which directly learns Declare constraints without any intermediate language.
Besides, this adopted approach is grounded on a SAT-based solver analogously to the works \cite{2018-Neider,2019-Camacho,2019-Riener}, where simple \ac{LTL} formulas are generated analysing a set of positive and negative examples.
%\cite{2018-Neider} resort to decision trees to improve the performance while dealing with large input logs; \cite{2019-Camacho} exploit \ac{AFA} to learn \ac{LTL} formulas consistent with positive and negative example sets; and \cite{2019-Riener} propose to split the search space into smaller subproblems using partial \acp{DAG}.

Our notion of negative example is similar to the definitions of syntactical and semantic noise of \cite{2009-Gunther}, because our approach is able to extract both the syntactic information that characterise the positive examples w.r.t. negative, and the relevant semantic difference between traces that has been partially or totally modified at a certain point in time.
%As a minor point, we also might notice that these works provides in output LTL formulas, while we opt for Declare formulas with LTL$_f$ semantics.

In this sense, our work is also closely related to deviance mining approaches \cite{2016-Nguyen}, i.e., techniques to extract the relevant details characterising those traces showing a sequence of activities that deviates from the expected behaviour. Wheres some deviance mining approaches \cite{2014-Suriadi,2014-Armas} focus on the differences between models discovered from deviant and non-deviant traces, others \cite{2013-Suriadi,2015-Partington,2013-Bose,2007-Lo,2016-Bernardi} intend it as a sort of sequence classification: to discover patterns of activities that distinguish different types of traces.

 
 
\btext{Add somewhere something on Discovery with DCR graphs}
% (https://dblp.uni-trier.de/pers/hd/s/Slaats:Tijs)} 

%https://link.springer.com/chapter/10.1007%2F978-3-030-21290-2_37
%
%https://link.springer.com/chapter/10.1007%2F978-3-642-32885-5_6
%http://ceur-ws.org/Vol-1021/paper_10.pdf
%
%negation of declare constraints:
%https://www.researchgate.net/publication/284570318_Patterns_for_a_Log-Based_Strengthening_of_Declarative_Compliance_Models
%
%process mining ibrido
%https://link.springer.com/article/10.1007/s13740-020-00112-9
