%!TEX root = ./main.tex

\section{Discovering Business Processes from Positive \& Negative Traces}
\label{sec:deviant}

%\todocdf{Forse possiamo aggiornare la notazione e usare P per le tracce positive, N per le tracce negative, M per il modello discovered o viceversa}

%\btext{CHiara G + Bolognesi: fare un sunto di deviant}
%Vorremo riprendere la parte tecnica del lavoro under revision, senza chiamarlo ``background'' perché non è mica ancora pubblicato e assodato...


Our approach is based on the \nd binary miner~\cite{deviant-tkde}, which, given two input sets of positive and negative examples, aims at extracting a model accepting all positive traces and rejecting all negative ones. %negatives. 
%This operation includes an abstraction step, so that the resulting model would allow to classify also unknown traces, not reported in the input log. Depending on the choices made during the discovery process, the learned model can``shape'' the set of unknown traces in different ways.
In this work, we enrich \nd with the possibility to express domain-dependent preferences on the discovered models. Therefore, we  report some definitions and explanations from~\cite{deviant-tkde} that are useful to understand %also
our approach.

\nd starts from a certain \emph{language bias}: given a set of \declare templates $D$ and a set of activities $A$, we indicate with $D[A]$ the set of all possible groundings of templates in $D$ w.r.t. $A$, i.e., all the constraints that can be built using activities in $A$.

We respectively denote with $L^+$ and $L^-$ the sets of positive and negative examples %, reported 
in the input event log. \nd starts by considering a, possibly empty, initial model $P$, that is a set of \declare constraints known to characterize the examples in $L^+$. The goal of \nd is to refine $P$ taking into account both the positive and the negative examples.

\begin{definition}{}\label{def:cand}
Given the initial model $P$, a candidate solution for the discovery task is any set of constraints $S\subseteq D[A]$ s.t.
\begin{enumerate*} [label=\textit{(\roman*)}]
  \item $P\subseteq S$;
  \item $\forall t\in L^+$ we have $t\models S$;
  \item S maximizes the set $\{t\in L^-\mid t\not\models S\}$.
\end{enumerate*}
\end{definition}

\declare templates can be organized into a hierarchy of \emph{subsumption} \cite{2017-DiCiccio} according to the logical implications derivable from their semantics. Consistently with this concept, we introduce the following definition of \emph{generality} relation between models.
\begin{definition}{}\label{def:subs}
A model $M\subseteq D[A]$ is more general than $M'\subseteq D[A]$ (written as $M \succeq M'$) when for any $t\in A^*$, $t\models M' \Rightarrow t\models M$ , and strictly more general (written as $M \succ M'$) if $M$ is more general than $M'$ and there exists $t'\in A^*$ s.t.\ $t'\not\models M'$ and $t'\models M$.
\end{definition}

\nd integrates the \emph{subsumption} rules introduced in \cite{2017-DiCiccio}, into the \emph{deductive closure operator}.

%\theoremstyle{definition}\label{def:closure}
\begin{definition}{}
Given a set $R$ of subsumption rules, a deductive closure operator is a function $cl_R: \mathcal{P}(D[A])\rightarrow\mathcal{P}(D[A])$ that associates any set $M \in D[A]$ with all the constraints that can be logically derived from $M$ by applying one or more deduction rules in $R$.
\end{definition}
For brevity, in the rest of the paper, we will omit the set $R$ and we will simply write $cl(M)$ to indicate the deductive closure of $M$. The complete set of employed deduction rules is available in the source code~\cite{zenodo:experiments}.\footnote{The file \texttt{declare\_rules.txt} can be found in the%within the
\texttt{data} directory.}%\todo{usiamo lo stesso oggetto in Zenodo (nuova versione) o ne creiamo uno totalmente nuovo?}




Conceptually, the \nd approach can be seen as a two-step procedure: in the first step, a set of candidate constraints is built, and then solutions are selected among subsets of candidates via an optimization algorithm.
%
The set of candidate constraints is composed of %by
those in $D[A]$ that accept all positive examples and reject at least a negative one. To build this set, \nd constructs a \emph{compatibles} set, i.e., the set of constraints that accept all traces in $L^+$: 
\begin{equation}
{compatibles(D[A], L^+)} = \{c\in D[A]~|~\forall t\in L^+,~ t\models c \} \\
\end{equation}
%
Then, it defines the \textit{\sheriff} function to associate to any trace $t$ in $L^-$ the constraints of \textit{compatibles} that reject %rejects 
$t$:
\begin{equation}
{\textit{\sheriff}}(t) = \{c\in {compatibles}~|~t\not\models c\} \\
\label{eq:sheriffs}
\end{equation}
%
The \textit{\sheriff} function is used to construct the set of all candidate constraints from which a discovered model is derived, i.e., the set $\mathcal{C} = \bigcup_{t\in L^-} \textit{\sheriff}(t)$ of all the constraints in $D[A]$ accepting all positive traces and rejecting at least one negative trace. The solution space is therefore:
\begin{equation}
  \mathcal{Z}=\{M\in\mathcal{P}(\mathcal{C})\mid \forall t\in L^-~t\not\models M\cup P \text{ or } {\textit{\sheriff}}(t) = \emptyset \}
\end{equation}
%according to a certain domain-independent criterion. 
Due to the fact that not all the pairs of negative and positive sets of traces can be perfectly separated using \declare~\cite{DBLP:conf/bpm/SlaatsDB21}, there can be traces in $L^-$ for which the ${\textit{\sheriff}}$ is empty, meaning that those traces cannot be excluded by any model that guarantees the acceptance of all the positive ones.

The second step of \nd uses an optimization strategy to identify the solutions; in~\cite{deviant-tkde}, two different criteria were taken into account: \emph{generality} (or conversely, \emph{specificity}), and \emph{simplicity}.
If the user is interested in the most general model, then \nd employs the closure operator $cl$ to select the models $S \in \mathcal{Z}$ with the less restrictive behavior.
If the user wants the simplest model, \nd looks for the solutions with minimal closure size. In case of ties, the solution with the minimal size is preferred.



