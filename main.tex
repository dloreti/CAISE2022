% This is samplepaper.tex, a sample chapter demonstrating the
% LLNCS macro package for Springer Computer Science proceedings;
% Version 2.20 of 2017/10/04
%
\documentclass[runningheads]{llncs}
%
%\usepackage[T1]{fontenc}
\usepackage{graphicx}
% Used for displaying a sample figure. If possible, figure files should
% be included in EPS format.
%
% If you use the hyperref package, please uncomment the following line
% to display URLs in blue roman font according to Springer's eBook style:
%\renewcommand\UrlFont{\color{blue}\rmfamily}


\usepackage{todonotes}[inline]
\newcommand{\todofc}[1]{\todo[backgroundcolor=yellow,size=\tiny]{FC: #1}}
\newcommand{\tododl}[1]{\todo[backgroundcolor=pink,size=\tiny]{DL: #1}}

\usepackage{hyperref}
\usepackage{array}
\usepackage{multirow}
\usepackage{xspace}
\usepackage{booktabs}
\usepackage{siunitx}
\usepackage[inline,shortlabels]{enumitem}
\usepackage{subcaption}
\usepackage{caption}
\usepackage{bold-extra}
\usepackage{rotating}
\usepackage{xcolor,colortbl}
\usepackage{booktabs}
\usepackage[nolist]{acronym}
%\usepackage{newtxtext}


\usepackage{amssymb,amsmath}
\newcommand{\nd}{\texttt{NegDis}\xspace}
\newcommand{\declare}{\texttt{Declare}\xspace}

\usepackage{longtable,tabulary}

\newcommand\ele[1]{\texttt{#1}}
\newcommand\rel[1]{\texttt{\textit{#1}}}
\newcommand\gr[1]{\textsc{#1}}
\newcommand\lb{\textsl{LB meta-model}\xspace}


\newcommand\btext[1]{{\color{blue}{#1}}}

\newcommand{\sheriff}{sheriffs}
\newcommand{\asprin}{\emph{ASPrin}\xspace}



%%%%%%%%%%%%%%%%%%%%%%%%%%%%%%%%%%%%%%%%%%%%%%%%%%%%%%%%%%%%%%%%%%%%%%%%
\begin{document}
%
%\title{A literature-based business process meta-model}
\title{Shape Your Process: Discovering Business Processes from Positive and Negative Traces Taking into Account User Preferences}
%\title{Digging into BPM meta-models: criticalities and ideas towards solutions}

%
\titlerunning{Shape Your Process}
% If the paper title is too long for the running head, you can set
% an abbreviated paper title here
%

\author{Federico Chesani\inst{1}, Chiara Di Francescomarino\inst{2}, Chiara Ghidini\inst{2}, Giulia Grundler\inst{1}, Daniela Loreti\inst{1}, Fabrizio Maria Maggi\inst{3}, Paola Mello\inst{1}, Marco Montali\inst{3}, Sergio Tessaris\inst{3}}

%\author{First Author\inst{1}\orcidID{0000-1111-2222-3333} \and
%Second Author\inst{2,3}\orcidID{1111-2222-3333-4444} \and
%Third Author\inst{3}\orcidID{2222--3333-4444-5555}}
%
\authorrunning{Chesani et al.}
% First names are abbreviated in the running head.
% If there are more than two authors, 'et al.' is used.
%
\institute{DISI - University of Bologna, Italy \and
Fondazione Bruno Kessler, Trento, Italy \and
Free University of Bozen/Bolzano, Italy\\
\email{federico.chesani@unibo.it}}
%
\maketitle              % typeset the header of the contribution
%
\begin{abstract}
% ADD ABSTRACT\tododl{page limit: 16pp}

Process Discovery techniques focus on learning a process model starting from a given set of logged traces. The majority of the approaches consider only one set of examples to learn from, i.e. the log itself. Some recent works instead advocated the usefulness of taking into account two different sets of traces (aka positive and negative examples), with the goal of learning a process model that allows to discriminate which trace belongs to which set.

%When discovering declarative process models, the aim also becomes to provide an insight/explanation of which are the differences for which a trace would be classified as belonging to one set or another. Sometimes, however, too many models might be available, thus nullifying the discovery effort. Some preference criteria would be helpful to guide the discovery process towards a model among the many.

When discovering declarative process models, the aim also becomes to provide an insight/explanation of which are the differences between the two sets. Sometimes, however, too many models might be available, thus nullifying the discovery effort. Some preference criteria would be helpful to guide the discovery process towards a model among the many. 

%In this work we leverage our previous approach to provide the possibility, from the user viewpoint, of specifying preferences over trace activities and Declare templates. Such preferences are used to guide the discovery process, so that the output model will include, if possible, the preferred constraints. The user, through the preferences, instructs the discovery algorithm for looking for certain models: in other terms, some user knowledge on the desired outcome is exploited in the discovery process.

In this work we leverage our previous approach to provide the possibility, from the user viewpoint, of specifying preferences over trace activities and Declare templates. Such preferences are used to guide the discovery process, so that the output model will include, if possible, the preferred constraints. In other terms, some user knowledge about the desired outcome is exploited in the discovery process. To implement such a process, we exploit a logic-based framework that provides a sound and formal meaning to the notion of user preferences.

\keywords{Process mining \and process discovery \and declarative process models \and deviant traces.}
\end{abstract}
%
%
%

%!TEX root = ./main.tex


\section{Introduction}
\label{sec:introduction}

\emph{Process discovery} is one of the most investigated process mining techniques \cite{2012-Aalst}. It deals with the automatic learning of a process model from a given set of logged traces, each one representing the digital footprint of the execution of a case.

If we focus on the way process discovery techniques see the model-extraction task, we can divide them into two broad categories \cite{2018-Ponce}). The first category is constituted by works that tackle the problem of process discovery with one-class supervised learning techniques (see, e.g., \cite{2004-Aalst,2003-Weijters,2007-Gunther,2010-Aalst}). These works are driven by the assumption that all available log traces are instances of the process to be discovered. The second category comprises works that intend model-extraction as a two-class supervised task (\cite{2006-Maruster,2009-Goedertier,2009-Chesani}), which is driven by the possibility of partitioning the log traces into two sets according to some business or domain-related criterion. Usually these sets are referred to as \emph{positive} and \emph{negative} examples \cite{2018-Ponce}, and the goal is to learn a model that characterises one set w.r.t. the other. Hereafter we refer to miners of the first and second category as \emph{unary} ands \emph{binary} miners, respectively. 

Traditionally, the vast majority of works in the process discovery spectrum have followed the first approach. Nonetheless, few recent works \cite{2018-Ponce,DBLP:conf/bpm/SlaatsDB21,deviant-arxiv} have highlighted the importance of addressing this challenge as a two-class supervised task with different motivations: first and foremost the actual existence of \emph{positive} and \emph{negative} examples in real use cases \cite{2018-Ponce,DBLP:conf/bpm/SlaatsDB21}; second, the need to balance \emph{accuracy} and \emph{recall} \cite{DBLP:conf/bpm/SlaatsDB21}; and the need to discover the process of a particular variant according to a domain-specific need (e.g., the process characterizing ``fast'' traces) against the one that characterizes other variants, thus using the concept of \emph{positive} and \emph{negative} as a way to distinguish between two classes of examples \cite{deviant-arxiv}.  

A problem that remains unsolved in process discovery in general and, in process discovery as a two-class supervised task in particular, is the need to select, among all possible discovered models, the ones that fit better the expectations of a user. This is somehow true for the traditional discovery of procedural and declarative models in a one-class supervised learning fashion, where the discovered model that accepts all the positive examples is usually too complex (e.g., too spaghetti like), and mechanisms are introduced to ``select'' different outputs. Examples of criteria for the selection can be the frequency of a certain behavior (e.g., an activity or a path), or the presence of certain modeling patterns (e.g., a specific Declare pattern).   
The problem becomes even more compelling when we approach process discovery as a two-class supervised task. In fact, as recently shown by \cite{DBLP:conf/bpm/SlaatsDB21}, perfect binary miners, able to discover models that accept all positive examples and none of the negative examples, do not necessarily exist.  
 
In this paper we address the problem of inserting user preferences in the discovery of declarative process models  as a two-class supervised task. In particular we start from a recent work \cite{deviant-arxiv} that introduces the \nd binary miner for the \declare \cite{2009-Aalst} modeling language and show how to extend it with preference mechanisms able to increase the role of the users in obtaining their preferred models and provide the following contributions: 
\begin{enumerate}[{(i)}]
	\item we retain the original preference mechanism of \nd, based on the notion of model subsumption, which enables to obtain models that vary in generality/specificity, or simplicity. These preference mechanisms are considered to be domain-independent as they work on general properties of the representation and do not affect the language used to represent the model. In other words, they do not introduce any language bias (or preference) in the representation of the process model (Section \ref{sec:deviant});
	\item we introduce two types of new preferences: the first one on the \declare patterns considered, and the second one on the activities appearing in the model. These new types of preferences introduce a clear language bias in the representation as they strongly affect the language that can be used to build the model, and are often grounded in domain dependent settings. Furthermore we show how to obtain new preferences by combining the two (Section \ref{sec:tool}); 
	\item we show how to extend \nd to incorporate the new preferences using an ASP (Answer Set Programing) approach via the ASPrin tool.\todo{Add reference.}   
	\item We report on exploratory experiments applying an instantiation of \nd  to the data sets of \cite{DBLP:conf/bpm/SlaatsDB21} comparing results to ?????? \todo{cosa vogliamo far vedere nella valutazione? la faccimao?}.
\end{enumerate}


cardinalità...

%!TEX root = ./main.tex

% \section{Motivating Examples\ldots\ or ``which type or preferences we would like to see''}
\section{Why preferences over discovered models?}
\label{sec:example}


Users look for discovering models for a variety of reasons. A common one is related to the need of having a description/explanation of the process. Other reasons might be, for example, the need for detecting process deviations or process drifts. Or, as in the case with \nd, users might be interested to understand, from a model viewpoint, what distinguishes one set of traces from another. 

Depending on the discovery technique and the target language, many alternative models might describe the same process. For example, both BPMN and Declare allow to describe the same process using different constructs or templates. However, not all the discovered models are equivalent\footnote{Roughly speaking, two models are \emph{equivalent} if they accept and reject the same traces. Such a notion of equivalence hints to the possibility that given two models $\mathsf{M_1}$ and $\mathsf{M_2}$, opting for the former or the latter will not change which traces will be accepted or rejected.}
, and even when they are equivalent, there could be too many models to choose from.

Hence, the availability of many models might, in turn, hinder the usefulness of the discovery approach: the user will need a criterion for selecting few models among the many discovered.
%
Preferences over discovered models are then a way for supporting the user's needs. In particular, we envisage three different types of preferences: preferences over single activities, preferences over single model templates, and possibly combinations of both.



\subsection{Preferences over the process activities}
\label{subsec:prefOverActivities}

A first type of preference among discovered models is strictly related to the application domain. We can reasonably imagine that, depending on the user's goals, models that focus more on certain activities might be preferable w.r.t. other models.

\begin{example}
\label{ex:prefOverActivities}
Let us consider the quite common ``loan scenario'', where a bank receives a request for a loan, evaluates it, and provides an answer. Let us suppose that process instances have been classified into two sets, for example using some unexplainable machine learning method. The bank employee will look then for a model that helps her to understand differences between the two sets. Of course, the employee will not directly look into the logs, that for the easy of understanding we can suppose to be as follows:
% Let us consider the following, fictitious example, where both the positive and the negative sets contain only one trace each:
%
\begin{align*}
P & = \{\ (\mathsf{loanRequest}, \mathsf{requestEval}, \mathsf{notifyOutcome})\ \} \\
N & = \{\ (\mathsf{requestEval}, \mathsf{loanRequest})\ \}
\end{align*}
%
where the positive example set $P$ contains only one trace (lasting three activities), and the negative example set $N$ contains a single trace as well.

\noindent An employee working in the marketing department has in mind the bank slogan ``we always answer our customers''. Hence, she would be surely interested in the \textsf{notifyOutcome} activity. She would specify such preference, and the discovery algorithm would reply her with two models both involving the preferred activity\footnote{Other models exist, of course, but for the sake of clarity we mention only these models.}:
%
\begin{align*}
\mathsf{M_1} & = \{ \mathsf{response(requestEval, notifyOutcome)}\} \\
\mathsf{M_2} & = \{ \mathsf{existence(notifyOutcome)}\}
\end{align*}
%
\qed
%
\end{example}

Generally speaking, being able to specify a preference for models that refer to specific activities allows the user to answer to the question \emph{``Is it possible to discriminate between two sets of traces by looking at certain activities?''}. The discovery process is then domain-driven: many models describe the process, but those ones that focus on certain domain aspects should be searched.




\subsection{Preferences over Declare templates}
\label{subsec:prefOverTemplates}

Process description languages like, e.g., BPMN and Declare, are quite rich in their expressiveness, and allow to describe a process using different constructs or templates. This lead to the availability of alternative models that could be equivalent or not. Unfortunately, even when restricting our attention to equivalent models only, it is easy to see that they might not carry the exact meaning.

% One of the reasons why many alternative models can be discovered starting from a log resides in the richness of the adopted process description language. For example, both BPMN and Declare allow to describe the same process using different constructs or templates.  However, not all the discovered models are equivalent, and even when they are equivalent, they might not carry the exact meaning. Roughly speaking, two models are \emph{equivalent} if they accept and reject the same traces. hints to the possibility that given two models $\mathsf{M_1}$ and $\mathsf{M_2}$, opting for the former or the latter will not change which traces will be accepted or rejected (by definition).

\paragraph{Case 1: Equivalent models.} Let us consider the simpler case where a discovery algorithm provides in output two equivalent models. If from a ``conformance viewpoint'' nothing changes, from a high-level viewpoint different models might bear subtle meaning distinctions, as shown in the following example.

\begin{example}
\label{ex:unaryVsBinary}
Let us suppose to have the following log, whose traces have been classified into two sets:
%
\begin{align*}
P & = \{\ (\mathsf{a}, \mathsf{b}),\ (\mathsf{b}, \mathsf{a})\ \} \\
N & = \{\ (\mathsf{a}),\ (\mathsf{b})\ \}
\end{align*}
%
Alternative models allowing to distinguish traces belonging to the two sets are:
\begin{align*}
\mathsf{M_1} & = \{ \mathsf{existence(a),existence(b)}\} \\
\mathsf{M_2} & = \{ \mathsf{existence(a), responded\_existence(a, b)}\} \\
\mathsf{M_3} & = \{ \mathsf{existence(b), responded\_existence(b, a)}\} \\
\mathsf{M_4} & = \{ \mathsf{existence(a), co\_existence(a, b)}\} \\
\mathsf{M_5} & = \{ \mathsf{existence(b), co\_existence(a, b)}\}
\end{align*}
%

From the logical viewpoint, models $\mathsf{M_1}$--$\mathsf{M_5}$ are equivalent. However, models $\mathsf{M_2}$--$\mathsf{M_5}$ suggests that there is a relation between activities \textsf{a} and \textsf{b}: indeed, the models contain a binary constraint, whose purpose is indeed to highlight relations between activities. Model $\mathsf{M_1}$  instead does not tell us anything about possible links between activities \textsf{a} and \textsf{b}, and a user might conclude that no relation exists between the two activities.
\qed
\end{example}

%It is highly debatable if models containing unary Declare templates are better or worse than models containing binary templates.
Declare binary templates, by their nature, suggest a link between activities. Hence, a discovery algorithm that would return models with relation constraints would emphasize such links. The user would be left with the burden of understanding if such links are mere coincidences or artifacts of the discovery technique, or if rather some new knowledge has been discovered about the process.

We can imagine scenarios where users prefer models containing the minimum number of binary templates, so as to not incur into the risk of perceiving in-existent relations. On the other hand, we can easily imagine also of situations where the user is actively looking for new relations. In both cases, preferences about which templates should be preferably included into a model would allow the user to adapt the discovery process to her needs.

Notice also that Example \ref{ex:unaryVsBinary} might mislead the reader to think that preferences over templates is a matter of unary vs. binary constraints only. This is not true, since equivalence is a logic property that stems from the interplay between all the constraints within each single model. Models with many binary constraints might be proved to be equivalent. Let us consider the following example:

\begin{example}
\label{ex:alternateVsResponseEquiv}
Let us consider the following log:
\begin{align*}
P & = \{\ (\mathsf{a}, \mathsf{b}),\ (\mathsf{a}, \mathsf{b}, \mathsf{c}),(\mathsf{a}, \mathsf{c}, \mathsf{b})\ \} \\
N & = \{\ (\mathsf{a}), (\mathsf{a}, \mathsf{c})\ \}
\end{align*}
%
Two alternative models that accept the postive examples and rejects the negative ones are:
\begin{align*}
\mathsf{M_1} & = \{ \mathsf{absence2(a),response(a,b)}\} \\
\mathsf{M_2} & = \{ \mathsf{absence2(a),alternate\_response(a, b)}\}
\end{align*}
Models $\mathsf{M_1}$ and $\mathsf{M_2}$ are equivalent due to the interplay of the constraint \textsf{absence2} with the the \textsf{response} and the \textsf{alternate\_response} constraints: roughly speaking, being activity \textsf{a} forbidden to appear more than once, the effects of the stricter constraint \textsf{alternate\_reponse} are nullified.
\qed
\end{example}



\paragraph{Case 2: Non-equivalent models.}
Let us consider then the more complex case where alternative models are discovered, and they are not equivalent. This happens because a log is usually a partial view of all the possible execution traces. Traces not appearing in the log might be classified differently by different models. Not-yet-seen traces are \emph{unknown} w.r.t. the classification, but different models would classify them in a different manner. Different models would \emph{shape the unknown} differently.
% How is it possible that a discovery algorithm outputs different models, all able to correctly discriminate between positive and negative examples, but such models being not equivalent?
% The issue stems from the fact that a log is usually a partial view of all the possible execution traces. Traces not appearing in the log might be classified differently by different models. Not-yet-seen traces are \emph{unknown} w.r.t. the classification, but different models would classify them in a different manner. Different models would \emph{shape the unknown} differently.

\begin{example}
\label{ex:unknownShapedDifferently}
Let us consider the following log:
\begin{align*}
P & = \{\ (\mathsf{a}, \mathsf{b}),\ (\mathsf{b}, \mathsf{a})\ \} \\
N & = \{\ (\mathsf{a})\ \}
\end{align*}
%
Alternative models allowing to distinguish traces belonging to the two sets are:
\begin{align*}
\mathsf{M_1} & = \{ \mathsf{existence(a),existence(b)}\} \\
\mathsf{M_2} & = \{ \mathsf{responded\_existence(a, b)}\}
\end{align*}
%
Let us consider then the trace $\mathsf{(b)}$, that was not recorded in the log. Model $\mathsf{M_1}$ would reject it, whereas model $\mathsf{M_2}$ would accept it.
\qed
\end{example}

Example \ref{ex:unknownShapedDifferently} shows how traces not appearing in the log used for the discovery might be then classified differently by the models. A preference elicitation mechanism would allow the user to decide how the not-yet-seen traces would be classified, in a restricting or in a broader way. Another example is the following:

\begin{example}
\label{ex:alternateVsResponse}
Let us consider the following log:
\begin{align*}
P & = \{\ (\mathsf{a}, \mathsf{b}),\ (\mathsf{a}, \mathsf{b}, \mathsf{c}),(\mathsf{a}, \mathsf{c}, \mathsf{b})\ \} \\
N & = \{\ (\mathsf{a}), (\mathsf{a}, \mathsf{c})\ \}
\end{align*}
%
Two alternative models that accept the postive examples and rejects the negative ones are:
\begin{align*}
\mathsf{M_1} & = \{ \mathsf{response(a,b)}\} \\
\mathsf{M_2} & = \{ \mathsf{alternate\_response(a, b)}\} \tag*{$\square$}
\end{align*}
%\qed
\end{example}

In Example \ref{ex:alternateVsResponse} both the models suffice to classify a trace into one or the other class. However, model $\mathsf{M_2}$ is \emph{stricter}, since it accepts less traces and rejects more traces than $\mathsf{M_1}$. A user might express his preferences for more stricter or more general models.
% \btext{Esempio 4, ancora sul perchè preferrenze sui template: vincoli binari contro vincoli binari}


\subsection{Preferences over both activities and templates}
\label{sub:prefOverBoth}


The third type of preference over process models is a straightforward combination of the preference types introduced in Subsections \ref{subsec:prefOverActivities} and \ref{subsec:prefOverTemplates}. Domain-related knowledge would drive the attention to certain activities, and preferences between the template types would allow to focus on certain relation types.


\begin{example}
\label{ex:prefOverBoth}
Let us consider again the ``loan scenario'' previously introduced, and the same log as well:
%
\begin{align*}
P & = \{\ (\mathsf{loanRequest}, \mathsf{requestEval}, \mathsf{notifyOutcome})\ \} \\
N & = \{\ (\mathsf{requestEval}, \mathsf{loanRequest})\ \}
\end{align*}
%
Let us assume now the viewpoint of an employee working in the internal auditing department. Given that the wrong execution order of certain activities might be a symptom of some fraud, the employee would like to focus the attention to templates of type \textsf{response} and/or \textsf{precedence}, and in particular to those constraints involving the \textsf{requestEval} activity. The discovery algorithm would exploit such preference by looking for models with the elicited features, and would provide in outpur:
%
\begin{align*}
\mathsf{M} & = \{ \mathsf{precedence(requestEval,loanRequest)}\} \tag*{$\square$}
\end{align*}
%\qed
\end{example}

Notice that Example \ref{ex:prefOverBoth} shares the exact same log as Example \ref{ex:prefOverActivities}. However, the output is completely different: the user preference is used to guide the search for a model, thus giving in output models that should be of greater interest for the user.

%!TEX root = ./main.tex

\section{Discovering Business Processes from Positive \& Negative Traces}
\label{sec:deviant}

%\todocdf{Forse possiamo aggiornare la notazione e usare P per le tracce positive, N per le tracce negative, M per il modello discovered o viceversa}

%\btext{CHiara G + Bolognesi: fare un sunto di deviant}
%Vorremo riprendere la parte tecnica del lavoro under revision, senza chiamarlo ``background'' perché non è mica ancora pubblicato e assodato...


Our approach is based on the \nd binary miner~\cite{deviant-tkde}, which, given two input sets of positive and negative examples, aims at extracting a model accepting all positive traces and rejecting all negative ones. %negatives. 
%This operation includes an abstraction step, so that the resulting model would allow to classify also unknown traces, not reported in the input log. Depending on the choices made during the discovery process, the learned model can``shape'' the set of unknown traces in different ways.
In this work, we enrich \nd with the possibility to express domain-dependent preferences on the discovered models. Therefore, we  report some definitions and explanations from~\cite{deviant-tkde} that are useful to understand %also
our approach.

\nd starts from a certain \emph{language bias}: given a set of \declare templates $D$ and a set of activities $A$, we indicate with $D[A]$ the set of all possible groundings of templates in $D$ w.r.t. $A$, i.e., all the constraints that can be built using activities in $A$.

We respectively denote with $L^+$ and $L^-$ the sets of positive and negative examples %, reported 
in the input event log. \nd starts by considering a, possibly empty, initial model $P$, that is a set of \declare constraints known to characterize the examples in $L^+$. The goal of \nd is to refine $P$ taking into account both the positive and the negative examples.

\begin{definition}{}\label{def:cand}
Given the initial model $P$, a candidate solution for the discovery task is any set of constraints $S\subseteq D[A]$ s.t.
\begin{enumerate*} [label=\textit{(\roman*)}]
  \item $P\subseteq S$;
  \item $\forall t\in L^+$ we have $t\models S$;
  \item S maximizes the set $\{t\in L^-\mid t\not\models S\}$.
\end{enumerate*}
\end{definition}

\declare templates can be organized into a hierarchy of \emph{subsumption} \cite{2017-DiCiccio} according to the logical implications derivable from their semantics. Consistently with this concept, we introduce the following definition of \emph{generality} relation between models.
\begin{definition}{}\label{def:subs}
A model $M\subseteq D[A]$ is more general than $M'\subseteq D[A]$ (written as $M \succeq M'$) when for any $t\in A^*$, $t\models M' \Rightarrow t\models M$ , and strictly more general (written as $M \succ M'$) if $M$ is more general than $M'$ and there exists $t'\in A^*$ s.t.\ $t'\not\models M'$ and $t'\models M$.
\end{definition}

\nd integrates the \emph{subsumption} rules introduced in \cite{2017-DiCiccio}, into the \emph{deductive closure operator}.

%\theoremstyle{definition}\label{def:closure}
\begin{definition}{}
Given a set $R$ of subsumption rules, a deductive closure operator is a function $cl_R: \mathcal{P}(D[A])\rightarrow\mathcal{P}(D[A])$ that associates any set $M \in D[A]$ with all the constraints that can be logically derived from $M$ by applying one or more deduction rules in $R$.
\end{definition}
For brevity, in the rest of the paper, we will omit the set $R$ and we will simply write $cl(M)$ to indicate the deductive closure of $M$. The complete set of employed deduction rules is available in the source code~\cite{zenodo:experiments}.\footnote{The file \texttt{declare\_rules.txt} can be found in the%within the
\texttt{data} directory.}%\todo{usiamo lo stesso oggetto in Zenodo (nuova versione) o ne creiamo uno totalmente nuovo?}




Conceptually, the \nd approach can be seen as a two-step procedure: in the first step, a set of candidate constraints is built, and then solutions are selected among subsets of candidates via an optimization algorithm.
%
The set of candidate constraints is composed of %by
those in $D[A]$ that accept all positive examples and reject at least a negative one. To build this set, \nd constructs a \emph{compatibles} set, i.e., the set of constraints that accept all traces in $L^+$: 
\begin{equation}
{compatibles(D[A], L^+)} = \{c\in D[A]~|~\forall t\in L^+,~ t\models c \} \\
\end{equation}
%
Then, it defines the \textit{\sheriff} function to associate to any trace $t$ in $L^-$ the constraints of \textit{compatibles} that reject %rejects 
$t$:
\begin{equation}
{\textit{\sheriff}}(t) = \{c\in {compatibles}~|~t\not\models c\} \\
\label{eq:sheriffs}
\end{equation}
%
The \textit{\sheriff} function is used to construct the set of all candidate constraints from which a discovered model is derived, i.e., the set $\mathcal{C} = \bigcup_{t\in L^-} \textit{\sheriff}(t)$ of all the constraints in $D[A]$ accepting all positive traces and rejecting at least one negative trace. The solution space is therefore:
\begin{equation}
  \mathcal{Z}=\{M\in\mathcal{P}(\mathcal{C})\mid \forall t\in L^-~t\not\models M\cup P \text{ or } {\textit{\sheriff}}(t) = \emptyset \}
\end{equation}
%according to a certain domain-independent criterion. 
Due to the fact that not all the pairs of negative and positive sets of traces can be perfectly separated using \declare~\cite{DBLP:conf/bpm/SlaatsDB21}, there can be traces in $L^-$ for which the ${\textit{\sheriff}}$ is empty, meaning that those traces cannot be excluded by any model that guarantees the acceptance of all the positive ones.

The second step of \nd uses an optimization strategy to identify the solutions; in~\cite{deviant-tkde}, two different criteria were taken into account: \emph{generality} (or conversely, \emph{specificity}), and \emph{simplicity}.
If the user is interested in the most general model, then \nd employs the closure operator $cl$ to select the models $S \in \mathcal{Z}$ with the less restrictive behavior.
If the user wants the simplest model, \nd looks for the solutions with minimal closure size. In case of ties, the solution with the minimal size is preferred.





%%!TEX root = ./main.tex

\section{Including Prefereces }
\label{sec:preferences}

Different types of preferences
(0. subsumption, a. cardinality, b. templates, c. activities)

Noi diremmo che la subsumption la vorremmo, perché non avrebbe senso avere modelli più complicati dello stretto necessario... ma siamo sicuri? Infatti:

PROBLEMA TECNICO: ma la subsumption quando l'applichiamo? Qui siamo un po' confusi... perché applicarla sempre potrebbe dare problemi, ad esempio se la subsumption ci butta via modelli che invece sarebbero stati scelti dalle preferenze... cioè la dimensione "modello generale/modello specifico" potrebbe essere ortogonale alle preferenze: ma se sono dimensioni ortogonali, in che ordine le applichiamo?

[Questa sezione è un po' da pensare...]


%!TEX root = ./main.tex

\section{Adding preferences to process discovery: an implementation through ASPrin}
\label{sec:tool}

From a first sight it looks like the ${\textit{\sheriff}}$ function includes all what we need to generate ``preferred'' models; that is, by selecting exactly one constraint from each ${\textit{\sheriff}}(t)\neq\emptyset$ for $t\in L^-$ according to some patterns. However, that's clearly not the case because constraints are not independent: e.g., some constraints are more general than others (i.e.\ they satisfy a bigger set of traces, see Def.~\ref{def:subs}), but there is also the case in which selecting two constraints implies the validity of a third one. For example, selecting $\mathsf{precedence(a,b)}$ and $\mathsf{response(a,b)}$ would be equivalent to select $\mathsf{succession(a,b)}$ as well.

For this reason we cannot use any combinatorial optimiser, but we need a system enabling some form of constraint propagation. In~\cite{deviant-tkde} an Answer Set Programming (ASP) has been used, by leveraging the underlying rule based formalism enabling propagation, and \emph{weak constraints}\footnote{Note that these constraints are not related to the Declare constraints mentioned above, they share the name because of their relation with mathematical logic. The same meaning overloading applies to the word ``model''.} for optimisation~\cite{asp-intro,clingo}. The encoding of the optimisation problem follows the \emph{Generate and Test} ASP paradigm where part of the rules select a candidate ASP model (e.g., a subset of $\mathcal{C}$) and a set of constraints filter only the relevant models (e.g., those ``rejecting'' all the negative examples). Weak constraints are used to assign a preference value to any ASP model; i.e., a violated weak constraint doesn't reject the model but assigns a penalty.

In~\cite{deviant-tkde}, simple weak constraints were used to implement subset based preferences; however, specifying more complex preferences between ASP models using constraints become unmanageble and error-prone. The \emph{ASPrin} tool~\cite{DBLP:conf/aaai/BrewkaD0S15} layers upon the \emph{clingo} ASP solver~\cite{clingo}, enabling the specification of complex preference relations through user-defined  types and their arguments. By leveraging this flexibility we can capture the different types of preferences we identified in Section~\ref{sec:example}.

\btext{Sergio? Illustrare i diversi tipi di preferenze, come si inseriscono nella tecnica base di deviant e l'implementazione in ASPrin. La mia proposta e' di are una sezione sola, ma si puo' anhe spezzare la parte che introduce le preferenze e la tecnica dall'implementazione specifica nel tool se riteniamo opportuno descrivere in dettaglio il tool.}

Different types of preferences
(0. subsumption, a. cardinality, b. templates, c. activities)

Noi diremmo che la subsumption la vorremmo, perché non avrebbe senso avere modelli più complicati dello stretto necessario... ma siamo sicuri? Infatti:

PROBLEMA TECNICO: ma la subsumption quando l'applichiamo? Qui siamo un po' confusi... perché applicarla sempre potrebbe dare problemi, ad esempio se la subsumption ci butta via modelli che invece sarebbero stati scelti dalle preferenze... cioè la dimensione "modello generale/modello specifico" potrebbe essere ortogonale alle preferenze: ma se sono dimensioni ortogonali, in che ordine le applichiamo?

[Questa sezione è un po' da pensare...]











%!TEX root = ./main.tex

\section{Evaluating the discovery}
\label{sec:evaluation}

% \btext{@all? (in particolare chiara DF, FAbrizio, Marco e sergio per pensarla). Pensare alla valutazione. Ho recuperato i due dataset dei danesi (vedi git). Se volete si possono usare questi, pero' uno dei due log in realta' e' un insieme di 215 logs di cui loro hanno fatto la media. Forse io ne sceglierei due o tre (es Dreyers Foundation.xes + uno o due dei 215) e farei la valutazione su questi. Anche perche' piu' che valutazione e' far vedere cosa esce perche' nn mi e' chiaro come valutiamo n termini di metriche l'output.}

In Section \ref{sec:example}, we introduced the preference types through simple toy-like examples. The interested reader, however, might wonder about the usability and efficacy of our approach when applied to real-life cases. We explored the applicability of our approach using %to
two real-life event logs, namely DREYERS (492 positive traces and 208 negative ones) and CERV (55 positive traces and 102 negatives traces).\footnote{For reproducibility purposes the source code is available in~\cite{zenodo:experiments}, the DREYERS event log can be found in~\cite{DBLP:conf/ssci/DeboisS15,DBLP:conf/bpm/SlaatsDB21}, while the CERV event log is a proprietary dataset.} In both cases, we were able to find ten models satisfying the given preferences in a computation time between 1 and 3 seconds, using a normally-equipped laptop.



\subsection{The DREYERS log}
\label{subsec:dreyers}

The DREYERS log describes the Dreyer Foundation’s processes pertaining to their support to legal and architectural projects, and it has been used in \cite{DBLP:conf/ssci/DeboisS15,DBLP:conf/bpm/SlaatsDB21}. Each application to request the Foundation's support goes through a pre-screen that can lead to an initial rejection. The remaining applications undergo a review, in which at least one of the reviewers must be a lawyer or an architect, depending on the application type. The review phase is followed by a board meeting, where applications to be supported are selected and eventually funded.
Two sets of log traces are available in the dataset: a positive one collecting executions that did not fail and a negative one representing executions that were reset due to a system failure. %process instances that execute properly have been labeled as the positive example set, while process instances that do not adhere to the proper execution path belong to the negative example set. %The labeling was already available, and we are not aware of the adopted criteria.
%Two sets of log traces are available: process instances that do not go through the pre-screen belong to the negative example set, whereas the ones that execute till the end of the procedure have been labeled as positive.

Using this dataset, we played a sort of ``investigation game'', and explored the hypothesis that the type of application (architect- or lawyer- type) might affect the process outcome.
%More generally, one might be interested in knowing if there are any differences in models that explicitly mention one of the two possible types of application.
%To this end, we first specified a preference for models containing activity \textsf{Lawyer Review}, and then a preference for models containing activity \textsf{Architect Review}.
\federico{
To this end, we initially specified a preference \texttt{good\_action(Lawyer Review)}, and later on a preference \texttt{good\_action(Architect Review)}.
}
%
In both cases, more than one model %several models
satisfying the preferences were found. However, the two sets of models are identical (except for the architect/lawyer activity), showing that the process is independent of the application domain.
We report an example of a model obtained when specifying the preference for models containing activity \textsf{Architect Review}:
%
%l'ordine di restituzione è molto diverso e i primi restituiti sono quelli con l'azione in danese, quindi ne ho scelti due volutamente diversi... è meglio metterli uguali con solo l'azione architect/lawyer diversa?

{\small{
\begin{align*}
%\mathsf{M_1} = \{ \ & \mathsf{ alternateresponse(Undo\ payment, First\ payout)} \\
M_1 = \{ \ & \mathsf{ alternateresponse(Undo\ payment, First\ payout)} \\
& \mathsf{ chainprecedence(Fill\ out\ application, Initial\ Rejection)} \\
& \mathsf{ choice(Round\ ends, Change\ phase\ to\ Abort)} \\
& \mathsf{ notchainsuccession(Receive\ final\ report, First\ payout)} \\
& \mathsf{ notchainsuccession(Change\ phase\ to\ Preparation, Approve\ application)} \\
& \mathsf{ notchainsuccession(Change\ phase\ to\ Preparation, Execute\ Pre\ decision)} \\
& \mathsf{ notchainsuccession(Set\ to\ Pre\ approved, Round\ Ends)} \\
& \mathsf{ notsuccession(Architect\ Review, Approval\ on\ to\ the\ board)} \ \} 
\end{align*}
}}

%%%The DREYERS log consists of 492 positive traces and 208 negative ones, 130 positive paths and 29 negative ones. NegDis is able to return models that satisfy the whole positive log and violate all but three negative traces (two paths). 
%%%For the models of the previous example we report the number of traces and paths violated by each constraint:

%%%Model M1:\\
\begin{table}[t!]
\label{table:resultsModelM1}

\begin{center}
\begin{scriptsize}
\scalebox{1}{
\begin{tabular}{lcc}
\toprule
%\hline
\textbf{Constraints} & \textbf{Traces \#} & \textbf{Variants \#} \\
\hline
alternateresponse(Undo payment, First payout) & 2 & 2\\
%\hline
chainprecedence(Fill out application, Initial Rejection) & 3 & 2\\
%\hline
choice(Round ends, Change phase to Abort) & 195 & 17 \\
%\hline
notchainsuccession(Receive final report, First payout) & 1 & 1 \\
%\hline
notchainsuccession(Change phase to Preparation, Approve application) & 1 & 1\\
%\hline
notchainsuccession(Change phase to Preparation, Execute Pre decision) & 2 & 2 \\
%\hline
notchainsuccession(Set to Pre approved, Round Ends) & 2 & 2\\
%\hline
notsuccession(Architect Review, Approval on to the board) & 1 & 1\\
\hline
Traces not ruled out by the model & 3 & 2 \\
\hline
Total & 208 & 30\\
\hline
\end{tabular}
}
\end{scriptsize}
\end{center}
\caption{Traces ruled out by each constraint of model $M_1$.}%$\mathsf{M_1}$.}
\end{table}
%
Notably, as shown in Table 1, this model is able to discriminate between positive and negative examples except for three negative traces (two variants), that cannot be ruled out without discarding also some positive examples.

We continued our investigation by focusing on %the very beginning of the process, and on the specific 
activity \textsf{Initial Rejection}. We report here one of the returned models: %Many models were found, among them the following one:

{\small{\begin{align*}
%\mathsf{M_2} = \{ \ &  \mathsf{ absence2(Initial\ rejection)} \\
M_2 = \{ \ &  \mathsf{ absence2(Initial\ rejection)} \\
& \mathsf{ choice(Round\ Ends, Applicant\ informed)} \\
& \mathsf{ notchainsuccession(Set\ to\ Pre\ approved, Round\ Ends)} \\
& \mathsf{ notchainsuccession(Receive\ final\ report, First\ payout)} \\
& \mathsf{ notchainsuccession(Change\ phase\ to\ Preparation, Approve\ application)} \\
& \mathsf{ notchainsuccession(Change\ phase\ to\ Preparation, Execute\ Pre\ decision)} \\
& \mathsf{ notsuccession(Lawyer\ Review, Change\ phase\ to\ review)} \\
& \mathsf{ response(Undo\ payment, First\ payout) } \ \} 
\end{align*}
}
}
Model $M_2$ %$\mathsf{M_2}$
highlights the fact that some negative traces can be distinguishable from the positive ones because of the repetition of \textsf{Initial Rejection}: some traces, indeed, reported the execution of the activity twice, thus indicating an attention point for the process analyst.

\chiara{Finally, we did compare the effect of discovering models with or without the two preferred activities. For this we asked \nd to extract 10 optimal\todo{Lasciamo optimal?} models with no preferences, 10 with the  \textsf{Architect Review} preference and 10 with the \textsf{Initial Rejection} preference, and we pairwise compared the models with no preference and the ones with a preferred activity. When imposing no preferences, activity \textsf{Architect Review} shows up in only 4 of the 10 discovered models. Imposing the usage of \textsf{Initial Rejection} is instead ``unnecessary'' (a posteriori), as this activity is also present in all 10 models discovered without specifying any preference.}

%%%Model M2:\\
%%%
%%%\begin{tabular}{|l|l|l|}
%%%\hline
%%%Constraint & Traces & Paths \\
%%%\hline
%%%absence2(Initial rejection) & 3 & 2\\
%%%\hline
%%%choice(Round ends, Change phase to Abort) & 195 & 17 \\
%%%\hline
%%%notchainsuccession(Receive final report, First payout) & 1 & 1 \\
%%%\hline
%%%notchainsuccession(Change phase to Preparation, Approve application) & 1 & 1\\
%%%\hline
%%%notchainsuccession(Change phase to Preparation, Execute Pre decision) & 2 & 2 \\
%%%\hline
%%%notchainsuccession(Set to Pre approved, Round Ends) & 2 & 2\\
%%%\hline
%%%notsuccession(Lawyer Review, Change phase to review) & 1 & 1\\
%%%\hline
%%%response(Undo payment, First payout) & 2 & 2\\
%%%\hline
%%%\end{tabular}
%%
%%
%%
%%%\btext{DA FINIRE DOPO AVER PARLATO CON GIULIA}



\subsection{Evaluation on the CERV log}
\label{subsec:cerv}

CERV is an event log that describes a process pertaining to the cervical cancer screening in an Italian screening center, and it has been used in previous works \cite{2007b-Lamma,deviant-tkde}. The screening program is composed of five phases, organized sequentially: screening planning, invitation management, first level test with pap-test, second level test with colposcopy (only if the first test is positive), and eventually biopsy (if the second test gives a positive response). Several subjects do not respect the planned protocol: e.g., subjects might not show up at the first test, even if they have chosen a time slot. Moreover, a number of subjects prefer to consult physicians they trust more, in case of a positive response. As it commonly happens in socio-technical systems, a large variety of process instances appear in the log, not all them being compliant with the protocol.
%
Hence, the traces have been labeled by a domain expert as belonging either to the positive or the negative set, depending on their compliance with the adopted protocol.

We investigated the log by eliciting two preferences over the \textsf{precedence} and \textsf{succession} templates:
\federico{
\begin{center}
\texttt{good\_constraint(precedence).}\\
\texttt{good\_constraint(succession).}
\end{center}
}
%.
The first two returned models are:

\begin{align*}
%\mathsf{M_1} = \{ \ &  \mathsf{ alternateresponse(send\ positive\ pap\ test\ result, take\ a\ colposcopy\ examination)} \\
M_1 = \{ \ &  \mathsf{ alternateresponse(send\ positive\ pap\ test\ result, take\ a\ colposcopy\ examination)} \\
& \mathsf{ chainprecedence(invite, take\ a\ pap\ test\ examination)} \\
& \mathsf{ exclusivechoice(send\ pap\ test\ sample, reject)} \\
& \mathsf{precedence(send\ colposcopy\ uncertain\ result, send\ biopsy\ sample)} \ \}  \\ \\
%\mathsf{M_2} = \{ \ &  \mathsf{ alternateresponse(send\ positive\ pap\ test\ result, take\ a\ colposcopy\ examination)} \\
M_2 = \{ \ &  \mathsf{ alternateresponse(send\ positive\ pap\ test\ result, take\ a\ colposcopy\ examination)} \\
& \mathsf{ chainprecedence(invite, take\ a\ pap\ test\ examination)} \\
& \mathsf{ exclusivechoice(send\ pap\ test\ sample, reject)} \\
& \mathsf{succession(send\ colposcopy\ uncertain\ result, send\ biopsy\ sample)} \ \}  
\end{align*}

In model $M_1$, the \textsf{precedence} constraint implies that if a biopsy is executed, then the colposcopy examination has provided an uncertain result before. The second model is identical to the first one, except for the constraint related %subjected
to our preference. Interestingly, the \textsf{succession} relates %is about
the same activities involved in %by
the \textsf{precedence} constraint in the first model. The difference between the two models lies in the logical relation between \textsf{precedence} and \textsf{succession}: a trace that violates the former will always violate the latter (but not vice versa% the opposite
). It is then up to the domain expert to prefer a stricter or a more general behavior.
%, knowing that in this case such a choice would not affect the resulting model.

\chiara{Finally, we did compare the effect of discovering models with or without the two preferred templates, by extracting 10 optimal\todo{Lasciamo optimal?} models with no template preferences, and 10 each with the \textsf{Precedence} and \textsf{Succession} preference respectively. Interestingly enough, imposing the \textsf{Precedence} preference results in being extremely useful in this scenario. In fact, none of the 10 models discovered with no preferred template did contain a \textsf{Precedence} pattern. Similarly with \textsf{Succession}, which appears in only 1 of the 10 models discovered without specifying any preference.}

%%%For the models of the previous example we report the number of traces and paths violated by each constraint:
%%%
%%%Model M1:\\
%%%
%%%\begin{tabular}{|l|l|l|}
%%%\hline
%%%Constraint & Traces & Paths \\
%%%\hline
%%%alternateresponse(send positive pap test result, take a colposcopy examination) & 4 & 1\\
%%%\hline
%%%chainprecedence(invite, take a pap test examination) & 3 & 2\\
%%%\hline
%%%exclusivechoice(send pap test sample, reject) & 94 & 2 \\
%%%\hline
%%%precedence(send colposcopy uncertain result, send biopsy sample) & 4 & 3 \\
%%%\hline
%%%\end{tabular}
%%%
%%%Model M2:\\
%%%
%%%\begin{tabular}{|l|l|l|}
%%%\hline
%%%Constraint & Traces & Paths \\
%%%\hline
%%%alternateresponse(send positive pap test result, take a colposcopy examination) & 4 & 1\\
%%%\hline
%%%chainprecedence(invite, take a pap test examination) & 3 & 2 \\
%%%\hline
%%%exclusivechoice(send pap test sample, reject) & 94 & 2 \\
%%%\hline
%%%succession(send colposcopy uncertain result, send biopsy sample) & 4 & 3\\
%%%\hline
%%%\end{tabular}

%%%A second example on this event log aims to demonstrate the important contribution of a negative preference, that is, a preference for the absence of a certain activity or template in the model. We applied this kind of preference on the “reject” action and we didn’t obtain any model satisfying the desired criterion. However, this result still offers a valuable information, as it guarantees that the “reject” activity is essential to distinguish between the positive and negative traces.

%%%In order to evaluate the performances of our approach we also report, for each preference of the previous examples, the time needed to retrieve different numbers of models.\\
%%%\begin{tabular}{|c|c|c|c|c|c|}
%%%\cline{3-6}
%%%\multicolumn{2}{c}{} & \multicolumn{4}{|c|}{Required Time (s)}\\
%%%\hline
%%%\multicolumn{2}{|c}{Log} & \multicolumn{2}{|c}{DREYERS} & \multicolumn{2}{|c|}{CERV}\\
%%%\hline
%%%\multicolumn{2}{|c|}{Preference} & architect activity & lawyer activity & precedence template & succession template\\
%%%\hline
%%%\multirow{5}*{Number of models} & 1 & 0.185 & 0.178 & 0.273 & 0.281\\
%%%\cline{2-6}
%%%& 5 & 0.285 & 0.439 & 0.646 & 0.666\\
%%%\cline{2-6}
%%%& 10 & 0.849 & 1.635 & 1.106 & 3.545\\
%%%\cline{2-6}
%%%& 20 & 2.938 & 3.517 & 3.066 & 18.159\\
%%%\cline{2-6}
%%%& 50 & 14.167 & 22.485 & 76.766 & 130.256\\
%%%\hline
%%%\end{tabular}
%%%
%%%This table shows that the execution time depends both on the log and the specific preference.





%!TEX root = ./main.tex

\section{Related Work}
\label{sec:related}

\btext{Bologna? Related work di deviant accorciati + i danesi + deviant?}



When process models are loosely-structured, procedural discovery could produce spaghetti-models~\cite{2012-Maggi,deviant-tkde}. In that case, declarative approaches are more suitable for the purpose because they allow to briefly list all the required or prohibited behaviours in the business process. 

Over the last decade, several works focused on declarative process discovery~\cite{2011-Maggi,2012-Maggi,2012-Schunselaar,2017-DiCiccio,2015-DiCiccio}.
In~\cite{2011-Maggi}, Maggi et al. propose to build a set of all possible candidate Declare constraints considering all the activities that appear in the log, translate the constraints in \ac{LTL}, and check them against the whole log until certain levels of recall and specificity are reached. 
Techniques to refine the business model excluding vacuously satisfied constraints are the focus of the subsequent works by Schunselaar et al. \cite{2012-Schunselaar} and Maggi et al. \cite{2012-Maggi}, whereas Di Ciccio et al. \cite{2017-DiCiccio} propose an approach to filter out frequent redundancies and inconsistencies. 
All the cited declarative approaches do not deal with negative examples. Nonetheless, interestingly from our point of view, in \cite{2015-DiCiccio}  the authors define among the metrics to guide the declarative discovery approach, besides support and confidence, also the interest factor for each constraint w.r.t. the log, and the possibility to include in the search space constraints on prohibited behaviours.

Negative examples are instead actively employed in the declarative discovery approaches \cite{2007-Lamma,2009-Chesani,2010-Bellodi,2016-Bellodi}, %, which are all based on the functioning principles of \ac{ICL} algorithm \cite{1995-DaRaedt} and thus intrinsically dependent to the availability of both negative and positive examples. 
eventually relaying for the evaluation on the output of synthetical log generators able to produce both positive and negative process cases \cite{2009-Goedertier,2014-Stocker,2010-Hee,2019-Chesani,2017-Chesani,2020-Loreti}.
The technique by Lamma et al. \cite{2007-Lamma,2007b-Lamma} learns integrity constraints expressed as logical formulas, and translates them into the equivalent DecSerFlow constructs \cite{2006-Aalst}. 
Bellodi et al. \cite{2010-Bellodi}, \cite{2016-Bellodi} employ the same approach and automatically convert the results into Markov Logic formulas---statistical relational learning is used to determine the weight of each formula.
Analogously, Chesani et al. \cite{2009-Chesani} propose to learn a set of SCIFF rules \cite{2008-Alberti} and translate them into ConDec constraints \cite{2006-Pesic}. The approach that we adopt in this work instead, is the one presented in \cite{deviant-tkde}, which directly learns Declare constraints without any intermediate language.
Besides, this adopted approach is grounded on a SAT-based solver analogously to the works \cite{2018-Neider,2019-Camacho,2019-Riener}, where simple \ac{LTL} formulas are generated analysing a set of positive and negative examples.
%\cite{2018-Neider} resort to decision trees to improve the performance while dealing with large input logs; \cite{2019-Camacho} exploit \ac{AFA} to learn \ac{LTL} formulas consistent with positive and negative example sets; and \cite{2019-Riener} propose to split the search space into smaller subproblems using partial \acp{DAG}.

Our notion of negative example is similar to the definitions of syntactical and semantic noise of \cite{2009-Gunther}, because our approach is able to extract both the syntactic information that characterise the positive examples w.r.t. negative, and the relevant semantic difference between traces that has been partially or totally modified at a certain point in time.
%As a minor point, we also might notice that these works provides in output LTL formulas, while we opt for Declare formulas with LTL$_f$ semantics.

In this sense, our work is also closely related to deviance mining approaches \cite{2016-Nguyen}, i.e., techniques to extract the relevant details characterising those traces showing a sequence of activities that deviates from the expected behaviour. Wheres some deviance mining approaches \cite{2014-Suriadi,2014-Armas} focus on the differences between models discovered from deviant and non-deviant traces, others \cite{2013-Suriadi,2015-Partington,2013-Bose,2007-Lo,2016-Bernardi} intend it as a sort of sequence classification: to discover patterns of activities that distinguish different types of traces.

 
 
\btext{Add somewhere something on Discovery with DCR graphs}
% (https://dblp.uni-trier.de/pers/hd/s/Slaats:Tijs)} 

%https://link.springer.com/chapter/10.1007%2F978-3-030-21290-2_37
%
%https://link.springer.com/chapter/10.1007%2F978-3-642-32885-5_6
%http://ceur-ws.org/Vol-1021/paper_10.pdf
%
%negation of declare constraints:
%https://www.researchgate.net/publication/284570318_Patterns_for_a_Log-Based_Strengthening_of_Declarative_Compliance_Models
%
%process mining ibrido
%https://link.springer.com/article/10.1007/s13740-020-00112-9


%!TEX root = ./main.tex

\section{Conclusions}
\label{sec:conclusions}

\btext{@all, Chiara G}

% \input{metamodel}
%
% \input{discussion}
%
% \input{solutions}
%
% \input{related_works}
%
%
% \input{conclusion}

%
% ---- Bibliography ----
%
% BibTeX users should specify bibliography style 'splncs04'.
% References will then be sorted and formatted in the correct style.
%
 \bibliographystyle{splncs04}
 \bibliography{deviant}
%

\begin{acronym}[PUB/SUB]
    \acro{AFA}{Alternating Finite Automata}
    \acro{ASP}{Answer Set Programming}
    \acro{BPM}{Business Process Management}
    \acro{DAG}{Directed Acyclic Graphs}
    \acro{DPML}{Declarative Process Model Learner}
    \acro{ICL}{Inductive Constraint Logic}
    \acro{ILP}{Inductive Logic Programming}
    \acro{LTL}{Linear Temporal Logic}
    \acro{XES}{eXtensible Event Stream}
\end{acronym}

\end{document}
