% This is samplepaper.tex, a sample chapter demonstrating the
% LLNCS macro package for Springer Computer Science proceedings;
% Version 2.20 of 2017/10/04
%
\documentclass[runningheads]{llncs}
%
%\usepackage[T1]{fontenc}
\usepackage{graphicx}
% Used for displaying a sample figure. If possible, figure files should
% be included in EPS format.
%
% If you use the hyperref package, please uncomment the following line
% to display URLs in blue roman font according to Springer's eBook style:
%\renewcommand\UrlFont{\color{blue}\rmfamily}


\usepackage{todonotes}[inline]
\newcommand{\todofc}[1]{\todo[backgroundcolor=yellow,size=\tiny]{FC: #1}}
\newcommand{\tododl}[1]{\todo[backgroundcolor=pink,size=\tiny]{DL: #1}}
\newcommand{\todocdf}[1]{\todo[color=blue!40]{#1}}

\usepackage{hyperref}
\usepackage{array}
\usepackage{multirow}
\usepackage{xspace}
\usepackage{booktabs}
\usepackage{siunitx}
\usepackage[inline,shortlabels]{enumitem}
\usepackage{subcaption}
\usepackage{caption}
\usepackage{bold-extra}
\usepackage{rotating}
\usepackage{xcolor,colortbl}
\usepackage{booktabs}
\usepackage[nolist]{acronym}
\usepackage{newtxtext} 


%\usepackage{newtxtext}

\newcommand\chiara[1]{{\color{blue}{#1}}\xspace}

\usepackage{amssymb,amsmath}
\newcommand{\nd}{\texttt{NegDis}\xspace}
\newcommand{\declare}{\texttt{Declare}\xspace}

\usepackage{longtable,tabulary}

\newcommand\ele[1]{\texttt{#1}}
\newcommand\rel[1]{\texttt{\textit{#1}}}
\newcommand\gr[1]{\textsc{#1}}
\newcommand\lb{\textsl{LB meta-model}\xspace}


%\newcommand\btext[1]{{\color{black}{#1}}}

\newcommand{\sheriff}{sheriffs}
\newcommand{\asprin}{\emph{ASPrin}\xspace}

\begin{acronym}[PUB/SUB]
    \acro{AFA}{Alternating Finite Automata}
    \acro{ASP}{Answer Set Programming}
    \acro{BPM}{Business Process Management}
    \acro{DAG}{Directed Acyclic Graphs}
    \acro{DPML}{Declarative Process Model Learner}
    \acro{ICL}{Inductive Constraint Logic}
    \acro{ILP}{Inductive Logic Programming}
    \acro{LTL}{Linear Temporal Logic}
    \acro{XES}{eXtensible Event Stream}
\end{acronym}

\linespread{0.985}

\def\paramx {\taskize{A}}
\def\paramy {\taskize{B}}
\def\paramz {\taskize{C}}
\def\paramw {\taskize{D}}
\def\letterx {\ensuremath{a}}
\def\lettery {\ensuremath{b}}
\def\letterz {\ensuremath{c}}
\def\letterw {\ensuremath{d}}
\newcommand{\taskize}[1] {\ensuremath{\scalebox{0.85}{\textsf{#1}}}}
\def\taska {\taskize{a}}
\def\taskb {\taskize{b}}
\def\taskc {\taskize{c}}
\def\taskd {\taskize{d}}
\def\taskf {\taskize{f}}
\def\taske {\taskize{e}}
\def\taski {\taskize{i}}
\def\taskj {\taskize{j}}
\def\taskh {\taskize{h}}
\def\taskg {\taskize{g}}

\def\ExiTxt {existence}
\def\PartTxt {Participation}
\def\AbseTxt {absence}
\def\UniqTxt {AtMostOne}
\def\InitTxt {Init}
\def\EndTxt {End}
\def\ResExTxt {responded\_existence}
\def\RespTxt {response}
\def\AltRespTxt {alternate\_response}
\def\AltRespTxtShort {Alt.Response}
\def\ChaRespTxt {ChainResponse}
\def\PrecTxt {precedence}
\def\AltPrecTxt {AlternatePrecedence}
\def\AltPrecTxtShort {Alt.Precedence}
\def\ChaPrecTxt {chain\_precedence}
\def\CoExiTxt {co\_existence}
\def\SuccTxt {Succession}
\def\AltSuccTxt {AlternateSuccession}
\def\AltSuccTxtShort {Alt.Succession}
\def\ChaSuccTxt {ChainSuccession}
\def\NotCoExiTxt {NotCoExistence}
\def\NotSuccTxt {not\_succession}
\def\NotChaSuccTxt {not\_chain\_succession}
%
\def\ExiTmp {\ensuremath{\textsc{\ExiTxt}}}
\def\PartTmp {\ensuremath{\textsc{\PartTxt}}}
\def\AbseTmp {\ensuremath{\textsc{\AbseTxt}}}
\def\UniqTmp {\ensuremath{\textsc{\UniqTxt}}}
\def\InitTmp {\ensuremath{\textsc{\InitTxt}}}
\def\EndTmp {\ensuremath{\textsc{\EndTxt}}}
\def\ResExTmp {\ensuremath{\textsc{\ResExTxt}}}
\def\RespTmp {\ensuremath{\textsc{\RespTxt}}}
\def\AltRespTmp {\ensuremath{\textsc{\AltRespTxt}}}
\def\ChaRespTmp {\ensuremath{\textsc{\ChaRespTxt}}}
\def\PrecTmp {\ensuremath{\textsc{\PrecTxt}}}
\def\AltPrecTmp {\ensuremath{\textsc{\AltPrecTxt}}}
\def\ChaPrecTmp {\ensuremath{\textsc{\ChaPrecTxt}}}
\def\CoExiTmp {\ensuremath{\textsc{\CoExiTxt}}}
\def\SuccTmp {\ensuremath{\textsc{\SuccTxt}}}
\def\AltSuccTmp {\ensuremath{\textsc{\AltSuccTxt}}}
\def\ChaSuccTmp {\ensuremath{\textsc{\ChaSuccTxt}}}
\def\NotCoExiTmp {\ensuremath{\textsc{\NotCoExiTxt}}}
\def\NotSuccTmp {\ensuremath{\textsc{\NotSuccTxt}}}
\def\NotChaSuccTmp {\ensuremath{\textsc{\NotChaSuccTxt}}}
%
\newcommand{\Exi}[2] {\ensuremath{\mathsf{\ExiTxt(#1,#2)}}}
\newcommand{\Part}[1] {\ensuremath{\textsc{\PartTxt}(#1)}}
\newcommand{\Abse}[2] {\ensuremath{\mathsf{\AbseTxt(#1,#2)}}}
\newcommand{\Uniq}[1] {\ensuremath{\textsc{\UniqTxt}(#1)}}
\newcommand{\Ini}[1] {\ensuremath{\textsc{\InitTxt}(#1)}}
\newcommand{\End}[1] {\ensuremath{\textsc{\EndTxt}(#1)}}
\newcommand{\ResEx}[2] {\ensuremath{\mathsf{\ResExTxt(#1,#2)}}}
\newcommand{\Resp}[2] {\ensuremath{\mathsf{\RespTxt(#1,#2)}}}
\newcommand{\AltRes}[2] {\ensuremath{\mathsf{\AltRespTxt(#1,#2)}}}
\newcommand{\AltResp}[2] {\ensuremath{\mathsf{\AltRespTxt(#1,#2)}}}
\newcommand{\AltRespShort}[2] {\ensuremath{\textsc{\AltRespTxtShort}(#1,#2)}}
\newcommand{\ChaResp}[2] {\ensuremath{\textsc{\ChaRespTxt}(#1,#2)}}
\newcommand{\ChaRes}[2] {\ensuremath{\textsc{\ChaRespTxt}(#1,#2)}}
\newcommand{\Prec}[2] {\ensuremath{{\mathsf{\PrecTxt}(#1,#2)}}}
\newcommand{\AltPrec}[2] {\ensuremath{\textsc{\AltPrecTxt}(#1,#2)}}
\newcommand{\AltPrecShort}[2] {\ensuremath{\textsc{\AltPrecTxtShort}(#1,#2)}}
\newcommand{\ChaPrec}[2] {\ensuremath{\mathsf{\ChaPrecTxt(#1,#2)}}}
\newcommand{\CoExi}[2] {\ensuremath{\mathsf{\CoExiTxt(#1,#2)}}}
\newcommand{\Succ}[2] {\ensuremath{\textsc{\SuccTxt}(#1,#2)}}
\newcommand{\AltSucc}[2] {\ensuremath{\textsc{\AltSuccTxt}(#1,#2)}}
\newcommand{\AltSuccShort}[2] {\ensuremath{\textsc{\AltSuccTxtShort}(#1,#2)}}
\newcommand{\ChaSucc}[2] {\ensuremath{\textsc{\ChaSuccTxt}(#1,#2)}}
\newcommand{\NotCoExi}[2] {\ensuremath{\textsc{\NotCoExiTxt}(#1,#2)}}
\newcommand{\NotSucc}[2] {\ensuremath{\mathsf{\NotSuccTxt(#1,#2)}}}
\newcommand{\NotChaSucc}[2] {\ensuremath{\mathsf{\NotChaSuccTxt(#1,#2)}}} 
%%%%%%%%%%%%%%%%%%%%%%%%%%%%%%%%%%%%%%%%%%%%%%%%%%%%%%%%%%%%%%%%%%%%%%%%
\begin{document}
%
%\title{A literature-based business process meta-model}
\title{Shape Your Process: Discovering Business Processes from Positive and Negative Traces Taking into Account \chiara{Expert} User Preferences}
%\title{Digging into BPM meta-models: criticalities and ideas towards solutions}

%
\titlerunning{Shape Your Process}
% If the paper title is too long for the running head, you can set
% an abbreviated paper title here
%

\author{Federico Chesani\inst{1}, Chiara Di Francescomarino\inst{2}, Chiara Ghidini\inst{2}, \\Giulia Grundler\inst{1}, Daniela Loreti\inst{1}, Fabrizio Maria Maggi\inst{3}, Paola Mello\inst{1}, \\Marco Montali\inst{3}, Sergio Tessaris\inst{3}}

%\author{First Author\inst{1}\orcidID{0000-1111-2222-3333} \and
%Second Author\inst{2,3}\orcidID{1111-2222-3333-4444} \and
%Third Author\inst{3}\orcidID{2222--3333-4444-5555}}
%
\authorrunning{Chesani et al.}
% First names are abbreviated in the running head.
% If there are more than two authors, 'et al.' is used.
%
\institute{DISI - University of Bologna, Italy \and
Fondazione Bruno Kessler, Trento, Italy \and
Free University of Bozen/Bolzano, Italy\\
%\email{federico.chesani@unibo.it}
}
%
\maketitle              % typeset the header of the contribution
%
\begin{abstract}
% ADD ABSTRACT\tododl{page limit: 16pp}

Process discovery techniques focus on learning a process model starting from a given set of logged traces. The majority of the discovery approaches, however, only consider one set of examples to learn from, i.e., the log itself. Some recent works on declarative process discovery, instead, advocated the usefulness of taking into account two different sets of traces (a.k.a.\ positive and negative examples), with the goal of learning a set of constraints that is able to discriminate which trace belongs to which set.
%When discovering declarative process models, the aim also becomes to provide an insight/explanation of which are the differences for which a trace would be classified as belonging to one set or another. Sometimes, however, too many models might be available, thus nullifying the discovery effort. Some preference criteria would be helpful to guide the discovery process towards a model among the many.
%When discovering declarative process models, the aim also becomes to provide an insight/explanation of which are the differences between the two sets. 
Sometimes, however, too many possible sets of constraints might be available, thus nullifying the discovery effort. Therefore, some preference criteria would be helpful to guide the discovery process towards a set of constraints among the many. 
%In this work we leverage our previous approach to provide the possibility, from the user viewpoint, of specifying preferences over trace activities and Declare templates. Such preferences are used to guide the discovery process, so that the output model will include, if possible, the preferred constraints. The user, through the preferences, instructs the discovery algorithm for looking for certain models: in other terms, some user knowledge on the desired outcome is exploited in the discovery process.
In this work, we present an approach for the discovery of declarative models providing the possibility, from the user viewpoint, of specifying preferences on activities and constraint templates to be used to build the final set of constraints. Such preferences are used to guide the discovery process, so that the output set will include, if possible, the preferred constraints, thus exploiting some \chiara{expert} knowledge about the desired outcome. The approach is grounded in a logic-based framework that provides a sound and formal meaning to the notion of \chiara{expert} preferences.

\keywords{Process mining \and process discovery \and declarative process models \and deviant traces.}
\end{abstract}
%
%
%

%!TEX root = ./main.tex


\section{Introduction}
\label{sec:introduction}

\emph{Process discovery} is one of the most investigated process mining techniques \cite{2012-Aalst}. It deals with the automatic learning of a process model from a given set of logged traces, each one representing the digital footprint of the execution of a case.

If we focus on the way process discovery techniques see the model-extraction task, we can divide them into two broad categories. %\cite{2018-Ponce,DBLP:conf/bpm/SlaatsDB21}. 
The first category is constituted by works that tackle the problem of process discovery with one-class supervised learning techniques (see, e.g., \cite{2010-Aalst,2004-Aalst,2007-Gunther,2003-Weijters,DBLP:conf/bpm/AalstMFG17}). These works are driven by the assumption that all available log traces are instances of the process to be discovered and constitute the wast majority of works in the process discovery spectrum. 
The second category comprises works that intend model-extraction as a two-class supervised task, which is driven by the possibility of partitioning the log traces into two sets according to some business or domain-related criteria. Usually, these sets are referred to as \emph{positive} and \emph{negative} examples, and the goal is to learn a model that characterizes one set w.r.t.\ the other. These works are traditionally less represented (see \cite{2009-Chesani,2009-Goedertier,2006-Maruster}). Nonetheless, few recent works \cite{deviant-tkde,2018-Ponce,DBLP:conf/bpm/SlaatsDB21} have highlighted the importance of performing model-extraction as a two-class supervised task with different motivations: first, the actual existence of \emph{positive} and \emph{negative} examples in real use cases \cite{2018-Ponce,DBLP:conf/bpm/SlaatsDB21}; second, the need to balance \emph{accuracy} and \emph{recall} \cite{DBLP:conf/bpm/SlaatsDB21}; and third, the need to discover a particular process variant (e.g., the process characterizing ``fast'' traces) against the one that characterizes other variants, thus using the labels \emph{positive} and \emph{negative} to distinguish between two classes of examples \cite{deviant-tkde}.  
Hereafter, we refer to miners of the first and second category as \emph{unary} and \emph{binary} miners, respectively. 

A problem that remains unsolved in process discovery, in general, and in binary miners, in particular, is the need to select, among all possible discovered models, the ones that fit better the expectations of \chiara{expert users, that is, users who are knowledgeable about the specific domain and because of this have specific desiderata and expectations}. This is true for the traditional discovery of procedural and declarative models, where the discovered model that accepts all the positive examples is usually too complex (e.g., too spaghetti like), and mechanisms are introduced to ``select'' specific behaviors. Examples of criteria for this selection can be the frequency of a certain element (e.g., an activity or a path), or the presence of certain modeling patterns (e.g., a specific declarative pattern).   
The problem becomes even more compelling when we approach process discovery as a two-class supervised task. In fact, as recently shown in \cite{DBLP:conf/bpm/SlaatsDB21}, perfect binary miners, able to discover models that accept all positive examples and none of the negative examples, do not necessarily exist. In such cases, many sub-optimal models can be returned, leading to the issue of identifying criteria for preferring one model or the other.
 
In this paper, we address the problem of inserting \chiara {expert user} preferences, \chiara{(hereafter expert preferences)} in the discovery of declarative process models as a two-class supervised task. We start from a recent work \cite{deviant-tkde} that introduces the \nd binary miner for the \declare modeling language \cite{DBLP:conf/edoc/PesicSA07} (introduced in Section \ref{sec:prel}). Being \nd based on the logic-based framework ASP \cite{asp-intro}, it provides a formal framework with a clear semantics that allows the users to ``prioritize'' the discovery results. In this work, we extend \nd by introducing the \asprin tool \cite{DBLP:conf/aaai/BrewkaD0S15}, so as to support the notion of \chiara{expert} preferences while remaining within the context of a formal, logic-based semantics. The following contributions are provided:
%
\begin{enumerate}[{(i)}]
    \item we introduce and motivate two types of \chiara{expert} preferences: the first one on the \declare patterns to be used in the discovery task, and the second one on the activities appearing in the output model. Moreover, we discuss also a third type of preference coming from the combination of the first two (Section \ref{sec:example}).
%These new types of preferences are then used to guide the search for a preferred model;	
	\item we extend the original mechanism of \nd (Section \ref{sec:deviant}), by incorporating the \asprin tool \cite{DBLP:conf/aaai/BrewkaD0S15} into it. This allows us to integrate within a single framework both the \chiara{expert} preferences, as well as the original \nd mechanism based on \emph{model subsumption} (that is treated as a preference as well). In this way, we retain the original ability of obtaining models that vary in generality/specificity, or simplicity.
% These preference mechanisms are considered to be domain-independent as they work on general properties of the representation and do not affect the language used to represent the model. In other words, they do not introduce any language bias (or preference) in the representation of the process model (Section \ref{sec:deviant}); 
	\item we provide some hints about the implementation (Section \ref{sec:tool});   
	\item we report on exploratory experiments applying an instantiation of \nd  to the data sets used in \cite{2007b-Lamma,DBLP:conf/bpm/SlaatsDB21} (Section \ref{sec:evaluation}).
\end{enumerate}
Related works (Sect. \ref{sec:related}) and final considerations (Sect. \ref{sec:conclusions}) conclude this work.

%%%\begin{enumerate}[{(i)}]
%%%	\item we retain the original preference mechanism of \nd, based on the notion of model subsumption, which enables to obtain models that vary in generality/specificity, or simplicity. These preference mechanisms are considered to be domain-independent as they work on general properties of the representation and do not affect the language used to represent the model. In other words, they do not introduce any language bias (or preference) in the representation of the process model (Section \ref{sec:deviant});
%%%	\item we introduce two types of new preferences: the first one on the \declare patterns considered, and the second one on the activities appearing in the model. These new types of preferences introduce a clear language bias in the representation as they strongly affect the language that can be used to build the model, and are often grounded in domain dependent settings. Furthermore we show how to obtain new preferences by combining the two (Section \ref{sec:tool}); 
%%%	\item we show how to extend \nd to incorporate the new preferences using an \ac{ASP} \cite{2008-Lifschitz} approach via the \asprin tool \cite{DBLP:conf/aaai/BrewkaD0S15}.   
%%%	\item We report on exploratory experiments applying an instantiation of \nd  to the data sets of \cite{DBLP:conf/bpm/SlaatsDB21} comparing results to ?????? \todo{cosa vogliamo far vedere nella valutazione? la faccimao?}.
%%%\end{enumerate}

%!TEX root = ./main.tex

\section{The modeling language}
\label{sec:prel}
The discovery approach we introduce in this paper is based on \declare. \declare is a language for describing declarative process models first introduced
in~\cite{DBLP:conf/edoc/PesicSA07}. A \declare model consists of a set of constraints applied to
(atomic) activities. Constraints are, in turn, based on templates. Templates are
abstract parameterized patterns and constraints are their concrete
instantiations on real activities.
Templates have a graphical representation and their semantics can be formalized using different logics, the main one being  LTL for finite traces, making them verifiable and executable.
Each constraint inherits the graphical representation and semantics from its
template.
The major benefit of using templates is that analysts do not have to be aware of
the underlying logic-based formalization to understand the models. They work
with abstract representations of templates, while the underlying formulas
remain hidden. Table~\ref{table:declare} summarizes the main \declare constructs used in this paper. The reader can refer to \cite{DBLP:conf/edoc/PesicSA07} for a full description
of the language.

%\renewcommand{\arraystretch}{1.2}
\begin{table} [h]
\centering
\scalebox{0.8}{%
\begin{tabular}{ l p{5cm}}
\toprule
\textbf{Template} & \textbf{Explanation} \\  
\midrule
%\multicolumn{2}{l}{Existence templates}\\
%\midrule
$\Exi{n}{\paramx}$ &
$\paramx$ occurs at least $n$ times 
\\
$\Abse{m+1}{\paramx}$ & 
$\paramx$ occurs at most $m$ times 
\\
%$\Ini{\paramx}$ &
%{\paramx} is the \emph{first} to occur 
%\\
%$\End{\paramx}$ &
%{\paramx} is the \emph{last} to occur
%\\
\midrule
%\multicolumn{2}{l}{Relation templates}\\
%\midrule
$\ResEx{\paramx}{\paramy}$ &
If {\paramx} occurs, then {\paramy} occurs \\
$\Resp{\paramx}{\paramy}$ &
If {\paramx} occurs, then {\paramy} occurs after {\paramx} 
\\
$\AltResp{\paramx}{\paramy}$ &
Each time {\paramx} occurs, then {\paramy} occurs afterwards, before {\paramx} recurs
\\
%$\ChaResp{\paramx}{\paramy}$ &
%Each time {\paramx} occurs, then {\paramy} occurs immediately after \\
$\Prec{\paramx}{\paramy}$ &
{\paramy} occurs only if preceded by {\paramx} \\
%$\AltPrec{\paramx}{\paramy}$ &
%Each time {\paramy} occurs, it is preceded by {\paramx} and no other {\paramy} can recur in between \\
$\ChaPrec{\paramx}{\paramy}$ &
Each time {\paramy} occurs, then {\paramx} occurs immediately before \\
\midrule
%\multicolumn{2}{l}{Mutual relation templates}\\
%\midrule
$\CoExi{\paramx}{\paramy}$ &
If {\paramy} occurs, then {\paramx} occurs, and vice versa \\
%$\Succ{\paramx}{\paramy}$ &
%{\paramx} occurs if and only if {\paramy} occurs after {\paramx} \\
%$\AltSucc{\paramx}{\paramy}$ &
%{\paramx} and {\paramy} occur if and only if the latter follows the former, and they alternate each other \\
%$\ChaSucc{\paramx}{\paramy}$ &
%{\paramx} and {\paramy} occur if and only if the latter immediately follows the former \\
%\midrule
%\multicolumn{2}{l}{Negative relation templates}\\
%\midrule
%$\NotCoExi{\paramx}{\paramy}$ &
%{\paramx} and {\paramy} never occur together \\
$\NotSucc{\paramx}{\paramy}$ &
{\paramx} never occurs before {\paramy} \\
$\NotChaSucc{\paramx}{\paramy}$ &
{\paramx} and {\paramy} occur if and only if the latter does not immediately follow the former
\\
\bottomrule
\end{tabular}
}
\vspace{0.3cm}
\caption{\declare templates}
\vspace{-0.8cm}
\label{table:declare}
\end{table} 



\section{Why preferences on the discovered models?}
\label{sec:example}


Users look for discovering models for a variety of reasons. A common one is related to the need of having a description/explanation of a process. Other reasons might be, for example, the need for detecting process deviations or process drifts. Or, as in the case of \nd, \chiara {expert} users might be interested in understanding, from a model viewpoint, what distinguishes one set of traces from another. 

Depending on the discovery technique and the target language, many alternative models might describe the same process. For example, both BPMN and \declare allow us to describe the same process using different constructs or templates. However, not all the discovered models are equivalent\footnote{Roughly speaking, two models are \emph{equivalent} if they accept and reject the same traces. Such a notion of equivalence hints to the possibility that given two models $M_1$ and $M_2$, %$\mathsf{M_1}$ and $\mathsf{M_2}$,
opting for the former or the latter will not change which traces will be accepted or rejected.}, and even when they are equivalent, there could be too many models to choose from.

Since the availability of many models might, in turn, hinder the usefulness of the discovery approach, the \chiara{expert} user would need a criterion for selecting few models among the many discovered.
%
Preferences on the discovered models represent then a way for prioritizing the discovered models based on the \chiara{expert} user's needs. In particular, we envisage three different types of preferences: preferences over activities, preferences over templates, and a combination of both.



\subsection{Preferences over process activities}
\label{subsec:prefOverActivities}

A first type of preferences on the discovered models is strictly related to the application domain. \chiara{Indeed,}  depending on the \chiara{expert} user's goals, models that focus more on certain activities might be preferable.

\begin{example}
\label{ex:prefOverActivities}
Let us consider the quite common ``loan scenario'', where a bank receives a request for a loan, evaluates it, and provides an answer. Let us assume that process instances have been classified into two sets, for example including successful and unsuccessful applications. The bank employee will look then for a model that helps her to understand the differences between the two sets. Of course, the employee will not directly look into the logs, which, for simplicity, we can suppose to be as follows:
% Let us consider the following, fictitious example, where both the positive and the negative sets contain only one trace each:
%
\begin{align*}
%P
L^{+}& = \{\ \langle \mathsf{loanRequest}, \mathsf{requestEval}, \mathsf{notifyOutcome} \rangle \ \} \\
%N
L^- & = \{\ \langle \mathsf{requestEval}, \mathsf{loanRequest} \rangle \ \}
\end{align*}
%
where the positive example set $L^+$ %$P$
contains only one trace (composed of
%lasting
 three activities), and the negative example set $L^-$ %$N$
 contains a single trace as well.

\noindent If the employee is an employee working in the marketing department, she could have in mind the bank slogan ``we always answer our customers''. Hence, she would be surely interested in the \textsf{notifyOutcome} activity. By specifying such preference, the discovery algorithm would return two models both involving the preferred activity\footnote{Other models exist, of course, but, for the sake of clarity, we only mention two of them.}:
%
\begin{align*}
%\mathsf{M_1} & = \{ \mathsf{response(requestEval, notifyOutcome)}\} \\
%\mathsf{M_2} & = \{ \mathsf{existence(notifyOutcome)}\}
M_1 & = \{ \mathsf{response(requestEval, notifyOutcome)}\} \\
M_2 & = \{ \mathsf{existence(notifyOutcome)}\}
\end{align*}
%
\qed
%
\end{example}

Generally speaking, being able to specify a preference for models that refer to specific activities allows chiara{expert} users to answer the question \emph{``Is it possible to discriminate between two sets of traces by looking at certain activities?''}. The discovery process becomes, in this way, domain-driven: many models describe the process, but only those ones that focus on certain domain aspects should be returned. %searched.




\subsection{Preferences over Declare templates}
\label{subsec:prefOverTemplates}

Process description languages like, e.g., BPMN and \declare, are quite rich in their expressiveness, and allow us to describe a process using different constructs or templates. This leads to the availability of alternative models that could be equivalent or not. Unfortunately, even when restricting our attention to equivalent models only, it is easy to see that they might not convey the information in exactly the same way to users.
% carry the exact meaning.

% One of the reasons why many alternative models can be discovered starting from a log resides in the richness of the adopted process description language. For example, both BPMN and Declare allow to describe the same process using different constructs or templates.  However, not all the discovered models are equivalent, and even when they are equivalent, they might not carry the exact meaning. Roughly speaking, two models are \emph{equivalent} if they accept and reject the same traces. hints to the possibility that given two models $\mathsf{M_1}$ and $\mathsf{M_2}$, opting for the former or the latter will not change which traces will be accepted or rejected (by definition).

\paragraph{Case 1: Equivalent models.} Let us consider first the %simpler
 case where a discovery algorithm provides as %in
 output two equivalent models. If from a ``conformance viewpoint'' nothing changes, from a high-level viewpoint different models might bear subtle meaning distinctions, as shown in the following example.

\begin{example}
\label{ex:unaryVsBinary}
Let us assume %suppose
 to have the following log, whose traces have been classified into two sets:
%
\begin{align*}
%P
L^+ & = \{\ \langle \mathsf{a}, \mathsf{b}\rangle,\ \langle \mathsf{b}, \mathsf{a} \rangle \ \} \\
%N
L^- & = \{\ \langle \mathsf{a}\rangle,\ \langle \mathsf{b} \rangle \ \}
\end{align*}
%
Alternative models allowing us to represent the traces that belong to $L^+$ %$P$
and exclude the ones that belong to $L^-$ %$N$
are:%distinguish traces belonging to the two sets are:
\begin{align*}
%\mathsf{M_1} & = \{ \mathsf{existence(a),existence(b)}\} \\
%\mathsf{M_2} & = \{ \mathsf{existence(a), responded\_existence(a, b)}\} \\
%\mathsf{M_3} & = \{ \mathsf{existence(b), responded\_existence(b, a)}\} \\
%\mathsf{M_4} & = \{ \mathsf{existence(a), co\_existence(a, b)}\} \\
%\mathsf{M_5} & = \{ \mathsf{existence(b), co\_existence(a, b)}\}
M_1 & = \{ \mathsf{existence(a),existence(b)}\} \\
M_2 & = \{ \mathsf{existence(a), responded\_existence(a, b)}\} \\
M_3 & = \{ \mathsf{existence(b), responded\_existence(b, a)}\} \\
M_4 & = \{ \mathsf{existence(a), co\_existence(a, b)}\} \\
M_5 & = \{ \mathsf{existence(b), co\_existence(a, b)}\}
\end{align*}
%

From a %the
 logical viewpoint, models $M_1$--$M_5$ %$\mathsf{M_1}$--$\mathsf{M_5}$
 are equivalent. However, models  $M_2$--$M_5$ %$\mathsf{M_2}$--$\mathsf{M_5}$
 suggest that what distinguishes the traces in $L^+$ from the traces in $L^-$  
%there
 is a relation between activities \textsf{a} and \textsf{b}: indeed, these models contain a binary constraint, whose purpose is to highlight a relation between these two activities. %indeed, the models contain a binary constraint, whose purpose is indeed to highlight relations between activities.
 Model $M_1$, %$\mathsf{M_1}$, 
 instead, does not tell us anything about possible links between activities \textsf{a} and \textsf{b}, and a user might conclude that no relation exists between them.%the two activities.
\qed
\end{example}

%It is highly debatable if models containing unary Declare templates are better or worse than models containing binary templates.
\declare binary templates, by their nature, suggest a link between activities. Hence, a discovery algorithm that would return models with relation constraints would emphasize such links. The user would be left with the burden of understanding if such links are mere coincidences or artifacts of the discovery technique, or if rather some new knowledge has been discovered about the process.

We can imagine scenarios where \chiara{expert} users prefer models containing the minimum number of binary templates, so as not to incur into the risk of perceiving in-existent relations. On the other hand, we can easily think about %imagine also
 situations where \chiara{an expert} user is actively looking for %new %perche` new?
 relations. In both cases, preferences on 
%about
 which \declare templates should be preferably included into a model would allow the \chiara{expert} user to tailor the discovery process to her needs.

Notice also that Example \ref{ex:unaryVsBinary} might mislead the reader to think that preferences over templates is a matter of unary vs. binary constraints only. This is not the case, since equivalence is a logic property that stems from the interplay between all the constraints within each single model. Models with many binary constraints might be proved to be equivalent, as shown in the following example.

\begin{example}
\label{ex:alternateVsResponseEquiv}
Let us consider the following log:
\begin{align*}
%P
L^+& = \{\ \langle \mathsf{a}, \mathsf{b} \rangle,\ \langle \mathsf{a}, \mathsf{b}, \mathsf{c} \rangle, \langle \mathsf{a}, \mathsf{c}, \mathsf{b} \rangle\ \} \\
%N
L^-& = \{\ \langle \mathsf{a} \rangle, \langle \mathsf{a}, \mathsf{c} \rangle \ \}
\end{align*}
%
Two alternative models that accept the positive examples and reject the negative ones are:
\begin{align*}
%\mathsf{M_1} & = \{ \mathsf{absence2(a),response(a,b)}\} \\
%\mathsf{M_2} & = \{ \mathsf{absence2(a),alternate\_response(a, b)}\}
M_1 & = \{ \mathsf{absence2(a),response(a,b)}\} \\
M_2 & = \{ \mathsf{absence2(a),alternate\_response(a, b)}\}
\end{align*}
Models $M_1$ and $M_2$  %$\mathsf{M_1}$ and $\mathsf{M_2}$
are equivalent due to the interplay of the constraint \textsf{absence2} with the the \textsf{response} and the \textsf{alternate\_response} constraints: roughly speaking, being activity \textsf{a} forbidden to appear more than once, the effects of the stricter constraint \textsf{alternate\_reponse} are nullified.
\qed
\end{example}



\paragraph{Case 2: Non-equivalent models.}
Let us consider now the case where alternative non-equivalent models are discovered.
%Let us consider then the more complex case where alternative models are discovered, and they are not equivalent.
This might happen because a log is usually a partial view of all the possible execution traces. Not-yet-seen traces are \emph{unknown} w.r.t. the classification, but different models could classify them in different manners. Different models would \emph{shape the unknown} differently.
% How is it possible that a discovery algorithm outputs different models, all able to correctly discriminate between positive and negative examples, but such models being not equivalent?
% The issue stems from the fact that a log is usually a partial view of all the possible execution traces. Traces not appearing in the log might be classified differently by different models. Not-yet-seen traces are \emph{unknown} w.r.t. the classification, but different models would classify them in a different manner. Different models would \emph{shape the unknown} differently.

\begin{example}
\label{ex:unknownShapedDifferently}
Let us consider the following log:
\begin{align*}
%P 
L^+ & = \{\ \langle \mathsf{a}, \mathsf{b} \rangle,\ \langle \mathsf{b}, \mathsf{a} \rangle \ \} \\
%N
L^- & = \{\ \langle \mathsf{a} \rangle \ \}
\end{align*}
%
Alternative models that accept the traces in $L^+$ and discard the ones in $L^-$ are:
%allowing to distinguish traces belonging to the two sets are:
\begin{align*}
%\mathsf{M_1} & = \{ \mathsf{existence(a),existence(b)}\} \\
%\mathsf{M_2} & = \{ \mathsf{responded\_existence(a, b)}\}
M_1 & = \{ \mathsf{existence(a),existence(b)}\} \\
M_2 & = \{ \mathsf{responded\_existence(a, b)}\}
\end{align*}
%
Let us consider then the trace $\langle \mathsf{b} \rangle$, that was not recorded in the log. Model $M_1$ %$\mathsf{M_1}$
would reject it, whereas model $M_2$ %$\mathsf{M_2}$
would accept it.
\qed
\end{example}

Example \ref{ex:unknownShapedDifferently} shows how traces not appearing in the log used for the discovery might be classified differently by the discovered models. A preference elicitation mechanism would allow the \chiara{expert} user to decide how the not-yet-seen traces would be classified, in a restricting or in a broader way. Another example is given below.

\begin{example}
\label{ex:alternateVsResponse}
Let us consider the following log:
\begin{align*}
%P
L^+ & = \{\ \langle \mathsf{a}, \mathsf{b} \rangle,\ \langle \mathsf{a}, \mathsf{b}, \mathsf{c} \rangle, \langle \mathsf{a}, \mathsf{c}, \mathsf{b} \rangle \ \} \\
%N
L^- & = \{\ \langle \mathsf{a} \rangle, \langle \mathsf{a}, \mathsf{c} \rangle \ \}
\end{align*}
%
Two non-equivalent models that accept the positive examples and reject the negative ones are:
\begin{align*}
%\mathsf{M_1} & = \{ \mathsf{response(a,b)}\} \\
%\mathsf{M_2} & = \{ \mathsf{alternate\_response(a, b)}\} \tag*{$\square$}
M_1 & = \{ \mathsf{response(a,b)}\} \\
M_2 & = \{ \mathsf{alternate\_response(a, b)}\} \tag*{$\square$}
\end{align*}
%\qed
\end{example}

%In Example \ref{ex:alternateVsResponse} both the models 
Both models in Example~\ref{ex:alternateVsResponse} 
suffice to classify a trace into one or the other class. However, model $M_2$ %$\mathsf{M_2}$
is \emph{stricter}, since it accepts less traces and rejects more traces than $M_1$. %$\mathsf{M_1}$.
A \chiara{expert} user might express her preference for stricter or more general models.
% \btext{Esempio 4, ancora sul perchè preferrenze sui template: vincoli binari contro vincoli binari}


\subsection{Preferences over both activities and templates}
\label{sub:prefOverBoth}

The third type of preferences on the discovered models is a straightforward combination of the preference types introduced in Subsections \ref{subsec:prefOverActivities} and \ref{subsec:prefOverTemplates}. Domain-related knowledge would drive the attention to certain activities, and preferences over templates would allow focusing on certain relation types.%between the template types would allow to focus on certain relation types.


\begin{example}
\label{ex:prefOverBoth}
Let us consider again the ``loan scenario'' and the log:%previously introduced, and the same log as well:
%
\begin{align*}
%P
L^+ & = \{\ \langle \mathsf{loanRequest}, \mathsf{requestEval}, \mathsf{notifyOutcome} \rangle \ \} \\
%N
L^- & = \{\ \langle \mathsf{requestEval}, \mathsf{loanRequest} \rangle \ \}
\end{align*}
%
Let us consider %assume
now the viewpoint of an employee working in the internal auditing department. Given that the wrong execution order of certain activities might be a symptom of some fraud, the employee would like to focus the attention over %to
templates of type \textsf{response} and/or \textsf{precedence}, and, in particular, over %to
those constraints involving the \textsf{requestEval} activity. The discovery algorithm would exploit such preference by looking for models with the elicited features, and would provide in output:
%
\begin{align*}
%\mathsf{M} & = \{ \mathsf{precedence(requestEval,loanRequest)}\}
M & = \{ \mathsf{precedence(requestEval,loanRequest)}\}
\tag*{$\square$}
\end{align*}
%\qed
\end{example}

Notice that Example \ref{ex:prefOverBoth} shares the exact same log as Example \ref{ex:prefOverActivities}. However, the output is completely different: the preferences are used, indeed, to guide the search for a model, which is of interest for the \chiara{expert} user.

%!TEX root = ./main.tex

\section{Discovering Business Processes from Positive \& Negative Traces}
\label{sec:deviant}

%\todocdf{Forse possiamo aggiornare la notazione e usare P per le tracce positive, N per le tracce negative, M per il modello discovered o viceversa}

%\btext{CHiara G + Bolognesi: fare un sunto di deviant}
%Vorremo riprendere la parte tecnica del lavoro under revision, senza chiamarlo ``background'' perché non è mica ancora pubblicato e assodato...


Our approach is based on \btext{the} \nd binary miner~\cite{deviant-tkde}, which, given two input sets of positive and negative examples, aims at extracting a model accepting all positive traces and rejecting all \btext{negative ones.} %negatives. 
%This operation includes an abstraction step, so that the resulting model would allow to classify also unknown traces, not reported in the input log. Depending on the choices made during the discovery process, the learned model can``shape'' the set of unknown traces in different ways.
In this work, we enrich \nd with the possibility to express domain-dependent preferences on the discovered models. Therefore, we need to report here some definitions and explanations from~\cite{deviant-tkde} that are useful to understand %also
our approach.

\nd start\btext{s} from a certain \emph{language bias}: given a set of \declare templates $D$ and a set of activities $A$, we indicate with $D[A]$ the set of all possible grounding\btext{s} of templates in $D$ w.r.t. $A$, i.e., all the constraints that can be built using activities in $A$.

We respectively denote with $L^+$ and $L^-$ the sets of positive and negative examples %, reported 
in the input event log. \nd starts by considering a --- possibly empty --- initial model $P$, that is a set of \declare constraints known to characterize the examples in $L^+$. The goal of \nd is to refine $P$ taking into account both the positive and the negative examples.

\begin{definition}{}\label{def:cand}
Given the initial model $P$, a candidate solution for the discovery task is any set of constraints $S\subseteq D[A]$ s.t.
\begin{enumerate*} [label=\textit{(\roman*)}]
  \item $P\subseteq S$;
  \item $\forall t\in L^+$ we have $t\models S$;
  \item S maximizes the set $\{t\in L^-\mid t\not\models S\}$.
\end{enumerate*}
\end{definition}

\declare templates can be organized into a hierarchy of \emph{subsumption} \cite{2017-DiCiccio} according to the logical implications derivable from their semantics. Consistently with this concept, we introduce the following definition of \emph{generality} relation between models.
\begin{definition}{}\label{def:subs}
A model $M\subseteq D[A]$ is more general than $M'\subseteq D[A]$ (written as $M \succeq M'$) when for any $t\in A^*$, $t\models M' \Rightarrow t\models M$ , and strictly more general (written as $M \succ M'$) if $M$ is more general than $M'$ and there exists $t'\in A^*$ s.t.\ $t'\not\models M'$ and $t'\models M$.
\end{definition}

\nd integrates the \emph{subsumption} rules introduced in \cite{2017-DiCiccio}, into the \emph{deductive closure operator}.

%\theoremstyle{definition}\label{def:closure}
\begin{definition}{}
Given a set $R$ of subsumption rules, a deductive closure operator is a function $cl_R: \mathcal{P}(D[A])\rightarrow\mathcal{P}(D[A])$ that associates any set $M \in D[A]$ with all the constraints that can be logically derived from $M$ by applying one or more deduction rules in $R$.
\end{definition}
For brevity, in the rest of the paper, we will omit the set $R$ and we will simply write $cl(M)$ to indicate the deductive closure of $M$. The complete set of employed deduction rules is available in the source code~\cite{zenodo:experiments}.\footnote{The file \texttt{declare\_rules.txt} \btext{can be found in the  }%within the
\texttt{data} directory.}%\todo{usiamo lo stesso oggetto in Zenodo (nuova versione) o ne creiamo uno totalmente nuovo?}




Conceptually, the \nd approach can be seen as a two-step procedure: in the first step, a set of candidate constraints is built, and then solutions are selected among subsets of candidates via an optimization algorithm.
%
The set of candidate constraints is composed \btext{of} %by
those in $D[A]$ that accept all positive examples and reject at least a negative one. To build this set, \nd constructs a \emph{compatibles} set, i.e., the set of constraints that accept all traces in $L^+$: 
\begin{equation}
{compatibles(D[A], L^+)} = \{c\in D[A]~|~\forall t\in L^+,~ t\models c \} \\
\end{equation}
%
Then, it defines the \textit{\sheriff} function to associate to any trace $t$ in $L^-$ the constraints of \textit{compatibles} that \btext{reject} %rejects 
$t$:
\begin{equation}
{\textit{\sheriff}}(t) = \{c\in {compatibles}~|~t\not\models c\} \\
\end{equation}
%
The \textit{\sheriff} function is used to construct the set of all candidate constraints from which a discovered model is derived, i.e., the set $\mathcal{C} = \bigcup_{t\in L^-} \textit{\sheriff}(t)$ of all the constraints in $D[A]$ accepting all positive traces and rejecting at least one negative trace. The solution space is therefore:
\begin{equation}
  \mathcal{Z}=\{M\in\mathcal{P}(\mathcal{C})\mid \forall t\in L^-~t\not\models M\cup P \text{ or } {\textit{\sheriff}}(t) = \emptyset \}
\end{equation}
%according to a certain domain-independent criterion. 
Due to the fact that not all the pairs of negative and positive sets of traces can be perfectly separated using \declare~\cite{DBLP:conf/bpm/SlaatsDB21}, there can be traces in $L^-$ for which the ${\textit{\sheriff}}$ is empty, meaning that those traces cannot be excluded by any model that guarantees the acceptance of all the positive ones.

The second step of \nd uses an optimization strategy to identify the solutions; in~\cite{deviant-tkde}, two different criteria were taken into account: \emph{generality} (or conversely, \emph{specificity}), and \emph{simplicity}.
If the user is interested in the most general model, then \nd employs the closure operator $cl$ to select the models $S \in \mathcal{Z}$ with the less restrictive behavior.
If the user wants the simplest model, \nd looks for the solutions with minimal closure size. In case of ties, the solution with the minimal size is preferred.





%%!TEX root = ./main.tex

\section{Including Prefereces }
\label{sec:preferences}

Different types of preferences
(0. subsumption, a. cardinality, b. templates, c. activities)

Noi diremmo che la subsumption la vorremmo, perché non avrebbe senso avere modelli più complicati dello stretto necessario... ma siamo sicuri? Infatti:

PROBLEMA TECNICO: ma la subsumption quando l'applichiamo? Qui siamo un po' confusi... perché applicarla sempre potrebbe dare problemi, ad esempio se la subsumption ci butta via modelli che invece sarebbero stati scelti dalle preferenze... cioè la dimensione "modello generale/modello specifico" potrebbe essere ortogonale alle preferenze: ma se sono dimensioni ortogonali, in che ordine le applichiamo?

[Questa sezione è un po' da pensare...]


%!TEX root = ./main.tex

\section{Adding preferences to process discovery: an implementation through ASPrin}
\label{sec:tool}

\btext{Sergio? Illustrare i diversi tipi di preferenze, come si inseriscono nella tecnica base di deviant e l'implementazione in ASPrin. La mia proposta e' di are una sezione sola, ma si puo' anhe spezzare la parte che introduce le preferenze e la tecnica dall'implementazione specifica nel tool se riteniamo opportuno descrivere in dettaglio il tool.}

Different types of preferences
(0. subsumption, a. cardinality, b. templates, c. activities)

Noi diremmo che la subsumption la vorremmo, perché non avrebbe senso avere modelli più complicati dello stretto necessario... ma siamo sicuri? Infatti:

PROBLEMA TECNICO: ma la subsumption quando l'applichiamo? Qui siamo un po' confusi... perché applicarla sempre potrebbe dare problemi, ad esempio se la subsumption ci butta via modelli che invece sarebbero stati scelti dalle preferenze... cioè la dimensione "modello generale/modello specifico" potrebbe essere ortogonale alle preferenze: ma se sono dimensioni ortogonali, in che ordine le applichiamo?

[Questa sezione è un po' da pensare...]











%!TEX root = ./main.tex

\section{Evaluating the discovery}
\label{sec:evaluation}

% \btext{@all? (in particolare chiara DF, FAbrizio, Marco e sergio per pensarla). Pensare alla valutazione. Ho recuperato i due dataset dei danesi (vedi git). Se volete si possono usare questi, pero' uno dei due log in realta' e' un insieme di 215 logs di cui loro hanno fatto la media. Forse io ne sceglierei due o tre (es Dreyers Foundation.xes + uno o due dei 215) e farei la valutazione su questi. Anche perche' piu' che valutazione e' far vedere cosa esce perche' nn mi e' chiaro come valutiamo n termini di metriche l'output.}

In Section \ref{sec:example} we motivated some type of preferences by means of simple toy-like examples. The interested reader however might wonder about the usability and efficacy of our approach w.r.t. real-life cases. Having this in mind, we explored the appliability of our approach to two real-life event logs, namely DREYERS (492 positive traces and 208 negative ones) and CERV (55 positive traces and 102 negatives traces).



\subsection{The DREYERS log}
\label{subsec:dreyers}

The DREYERS log describes the Dreyer Foundation’s processes, regarding their support to legal and architectural projects and applications, and it has been used in related works \cite{DBLP:conf/ssci/DeboisS15,DBLP:conf/bpm/SlaatsDB21}. In the Dreyers process, each application that aims to obtain the Foundation's support goes through a pre-screen that can lead to an initial rejection. The remaining applications undergo a review, in which at least one of the reviewers must be a lawyer or an architect, depending on the type of application. The review phase is followed by a board meeting, where the Foundation decides which applications to support: in case of success the application is granted a payout.

As mentioned in \cite{DBLP:conf/bpm/SlaatsDB21}, two sets of log traces are available: process instances that execute properly have been labeled as the positive example set, while log traces whose process instances do not adhere to the proper execution path belong to the negative example set. The labeling of the traces was already available in the log, and we are not aware of the criteria used in the labeling.

Hence, we decided to play a sort of ``investigation game'', and to explore the hypothesis that indeed the type of application (architect-type or lawyer type) might affect the process outcome. More generally, one might be interested in knowing if there are any differences in models that explicitly mention one of the two possible types of application, so as to understand the possible variations in the process, depending on the legal or architectural context. To distinguish (traces referring to) the two types of applications, we resorted to focus on two activities that are peculiar of each application type, namely the “Lawyer Review” activity and the “Architect Review” activity.

In both cases several models satisfying the preferences were found. However, the two sets of models are identical (except for the architect/lawyer activity), showing that the process is probably independent of the application domain, as expected from the process description.
We report an example of a model obtained when specifying a preference for the “Architect Review” activity:

%l'ordine di restituzione è molto diverso e i primi restituiti sono quelli con l'azione in danese, quindi ne ho scelti due volutamente diversi... è meglio metterli uguali con solo l'azione architect/lawyer diversa?

\begin{align*}
\mathsf{M_1} = \{ & \\
& \mathsf{ alternateresponse(Undo\ payment, First\ payout)} \\
& \mathsf{ chainprecedence(Fill\ out\ application, Initial\ Rejection)} \\
& \mathsf{ choice(Round\ ends, Change\ phase\ to\ Abort)} \\
& \mathsf{ notchainsuccession(Receive\ final\ report, First\ payout)} \\
& \mathsf{ notchainsuccession(Change\ phase\ to\ Preparation, Approve\ application)} \\
& \mathsf{ notchainsuccession(Change\ phase\ to\ Preparation, Execute\ Pre\ decision)} \\
& \mathsf{ notchainsuccession(Set\ to\ Pre\ approved, Round\ Ends)} \\
& \mathsf{ notsuccession(Architect\ Review, Approval\ on\ to\ the\ board)} \\
\} & % \\
%%%\mathsf{M_2} = \{ & \\
%%%& \mathsf{ absence2(Initial\ rejection)} \\
%%%& \mathsf{ choice(Round\ Ends, Applicant\ informed)} \\
%%%& \mathsf{ notchainsuccession(Set\ to\ Pre\ approved, Round\ Ends)} \\
%%%& \mathsf{ notchainsuccession(Receive\ final\ report, First\ payout)} \\
%%%& \mathsf{ notchainsuccession(Change\ phase\ to\ Preparation, Approve\ application)} \\
%%%& \mathsf{ notchainsuccession(Change\ phase\ to\ Preparation, Execute\ Pre\ decision)} \\
%%%& \mathsf{ notsuccession(Lawyer\ Review, Change\ phase\ to\ review)} \\
%%%& \mathsf{ response(Undo\ payment, First\ payout) } \\
%%%\} & 
\end{align*}

%%%The DREYERS log consists of 492 positive traces and 208 negative ones, 130 positive paths and 29 negative ones. NegDis is able to return models that satisfy the whole positive log and violate all but three negative traces (two paths). 
%%%For the models of the previous example we report the number of traces and paths violated by each constraint:

%%%Model M1:\\
\begin{table}
\label{table:resultsModelM1}
\begin{center}
\begin{scriptsize}
\begin{tabular}{lcc}
%\hline
Constraint & Traces & Paths \\
\hline
alternateresponse(Undo payment, First payout) & 2 & 2\\
%\hline
chainprecedence(Fill out application, Initial Rejection) & 3 & 2\\
%\hline
choice(Round ends, Change phase to Abort) & 195 & 17 \\
%\hline
notchainsuccession(Receive final report, First payout) & 1 & 1 \\
%\hline
notchainsuccession(Change phase to Preparation, Approve application) & 1 & 1\\
%\hline
notchainsuccession(Change phase to Preparation, Execute Pre decision) & 2 & 2 \\
%\hline
notchainsuccession(Set to Pre approved, Round Ends) & 2 & 2\\
%\hline
notsuccession(Architect Review, Approval on to the board) & 1 & 1\\
\hline
Traces not ruled out by the model & 3 & 2 \\
\hline
Total & 208 & 30\\
\hline
\end{tabular}
\end{scriptsize}
\end{center}
\caption{Traces ruled out by model $\mathsf{M_1}$.}
\end{table}

Notably, the obtained model is able to discriminate between positive and negative examples except for three negative traces (two paths), that cannot be ruled out without discarding also some positive examples.

We continued our investigation by focusing on the very beginning of the process, and on the specific activity \textsf{Initial Rejection}. Our approach provided different many models, among them the following one:

\begin{align*}
\mathsf{M_2} = \{ & \\
& \mathsf{ absence2(Initial\ rejection)} \\
& \mathsf{ choice(Round\ Ends, Applicant\ informed)} \\
& \mathsf{ notchainsuccession(Set\ to\ Pre\ approved, Round\ Ends)} \\
& \mathsf{ notchainsuccession(Receive\ final\ report, First\ payout)} \\
& \mathsf{ notchainsuccession(Change\ phase\ to\ Preparation, Approve\ application)} \\
& \mathsf{ notchainsuccession(Change\ phase\ to\ Preparation, Execute\ Pre\ decision)} \\
& \mathsf{ notsuccession(Lawyer\ Review, Change\ phase\ to\ review)} \\
& \mathsf{ response(Undo\ payment, First\ payout) } \\
\} & 
\end{align*}
%
Model $\mathsf{M_2}$ highlights the fact that some negative traces can be distinguishable from the positive ones because of the repetition of the \textsf{Initial Rejection}: indeed, some traces reported the execution of the activity twice, thus indicating an attention point for the process manager.

%%%Model M2:\\
%%%
%%%\begin{tabular}{|l|l|l|}
%%%\hline
%%%Constraint & Traces & Paths \\
%%%\hline
%%%absence2(Initial rejection) & 3 & 2\\
%%%\hline
%%%choice(Round ends, Change phase to Abort) & 195 & 17 \\
%%%\hline
%%%notchainsuccession(Receive final report, First payout) & 1 & 1 \\
%%%\hline
%%%notchainsuccession(Change phase to Preparation, Approve application) & 1 & 1\\
%%%\hline
%%%notchainsuccession(Change phase to Preparation, Execute Pre decision) & 2 & 2 \\
%%%\hline
%%%notchainsuccession(Set to Pre approved, Round Ends) & 2 & 2\\
%%%\hline
%%%notsuccession(Lawyer Review, Change phase to review) & 1 & 1\\
%%%\hline
%%%response(Undo payment, First payout) & 2 & 2\\
%%%\hline
%%%\end{tabular}
%%
%%
%%
%%%\btext{DA FINIRE DOPO AVER PARLATO CON GIULIA}


\subsection{Evaluation on the CERV log}
\label{subsec:cerv}

CERV is an event log that describes the process of cervical cancer screening in an Italian screening center, and it has been used in previous related works \cite{2007b-Lamma,deviant-tkde}. The screening program is composed of five phases, organized sequentially: screening planning, invitation management, first level test with pap-test, second level test with colposcopy (only if the first test is positive), and eventually biopsy (if the second test gives a positive response). A number of subjects do not respect the planned protocol: for example, subjects do not show up at the first test, even if they chose the temporal slot themselves. Moreover, a number of subjects prefer to consult other physicians they trust more, if the second or the third test gave a positive response. All these cases provided a great variety of process instances that, from the viewpoint of the manager, are not compliant with the intended protocol, as it commonly happens in socio-technical systems.
%
Hence, the traces have been labeled by a domain expert as belonging either to the positive or the negative set, depending on their compliance to the adopted protocol.

To investigate 
%All models returned by NegDis satisfy the whole positive log and violate all the negative traces. 
In this example, we applied the preferences approach over the “precedence” and “succession” Declare templates. The two first resulting models for, respectively, precedence and succession are as follows:

\begin{align*}
\mathsf{M_1} = \{ & \\
& \mathsf{ alternateresponse(send\ positive\ pap\ test\ result, take\ a\ colposcopy\ examination)} \\
& \mathsf{ chainprecedence(invite, take\ a\ pap\ test\ examination)} \\
& \mathsf{ exclusivechoice(send\ pap\ test\ sample, reject)} \\
& \mathsf{precedence(send\ colposcopy\ uncertain\ result, send\ biopsy\ sample)} \\
\} & \\
\mathsf{M_2} = \{ & \\
& \mathsf{ alternateresponse(send\ positive\ pap\ test\ result, take\ a\ colposcopy\ examination)} \\
& \mathsf{ chainprecedence(invite, take\ a\ pap\ test\ examination)} \\
& \mathsf{ exclusivechoice(send\ pap\ test\ sample, reject)} \\
& \mathsf{succession(send\ colposcopy\ uncertain\ result, send\ biopsy\ sample)} \\
\} & 
\end{align*}

%M1: { alternateresponse(send positive pap test result, take a colposcopy examination), 
%chainprecedence,invito(take a pap test examination), 
%exclusivechoice(send pap test sample reject),
%precedence(send colposcopy uncertain result, send biopsy sample) }

%M2: { alternateresponse(send positive pap test result, take a colposcopy examination),
%chainprecedence,invito(take a pap test examination),
%exclusivechoice(send pap test sample, reject),
%succession(send colposcopy uncertain result, send biopsy sample) }

These are identical except for the constraints targeted by the preferences, that give a slightly different meaning to the models. In the first model, the precedence constraint implies that a biopsy is executed only if the colposcopy examination has an uncertain result, while the succession constraint in the second model also entails that the biopsy is always executed in those circumstances, giving an additional information.
However, the second model would not exist without the preference expression, because “precedence” subsumes “succession” and the generality criterion would prune it.
%la sussunzione e il criterio di generalità li assumo già spiegati, vero? 

For the models of the previous example we report the number of traces and paths violated by each constraint:

Model M1:\\

\begin{tabular}{|l|l|l|}
\hline
Constraint & Traces & Paths \\
\hline
alternateresponse(send positive pap test result, take a colposcopy examination) & 4 & 1\\
\hline
chainprecedence(invite, take a pap test examination) & 3 & 2\\
\hline
exclusivechoice(send pap test sample, reject) & 94 & 2 \\
\hline
precedence(send colposcopy uncertain result, send biopsy sample) & 4 & 3 \\
\hline
\end{tabular}

Model M2:\\

\begin{tabular}{|l|l|l|}
\hline
Constraint & Traces & Paths \\
\hline
alternateresponse(send positive pap test result, take a colposcopy examination) & 4 & 1\\
\hline
chainprecedence(invite, take a pap test examination) & 3 & 2 \\
\hline
exclusivechoice(send pap test sample, reject) & 94 & 2 \\
\hline
succession(send colposcopy uncertain result, send biopsy sample) & 4 & 3\\
\hline
\end{tabular}


A second example on this event log aims to demonstrate the important contribution of a negative preference, that is, a preference for the absence of a certain activity or template in the model. We applied this kind of preference on the “reject” action and we didn’t obtain any model satisfying the desired criterion. However, this result still offers a valuable information, as it guarantees that the “reject” activity is essential to distinguish between the positive and negative traces.



In order to evaluate the performances of our approach we also report, for each preference of the previous examples, the time needed to retrieve different numbers of models.\\
\begin{tabular}{|c|c|c|c|c|c|}
\cline{3-6}
\multicolumn{2}{c}{} & \multicolumn{4}{|c|}{Required Time (s)}\\
\hline
\multicolumn{2}{|c}{Log} & \multicolumn{2}{|c}{DREYERS} & \multicolumn{2}{|c|}{CERV}\\
\hline
\multicolumn{2}{|c|}{Preference} & architect activity & lawyer activity & precedence template & succession template\\
\hline
\multirow{5}*{Number of models} & 1 & 0.185 & 0.178 & 0.273 & 0.281\\
\cline{2-6}
& 5 & 0.285 & 0.439 & 0.646 & 0.666\\
\cline{2-6}
& 10 & 0.849 & 1.635 & 1.106 & 3.545\\
\cline{2-6}
& 20 & 2.938 & 3.517 & 3.066 & 18.159\\
\cline{2-6}
& 50 & 14.167 & 22.485 & 76.766 & 130.256\\
\hline
\end{tabular}

This table shows that the execution time depends both on the log and the specific preference.



\subsection{VECCHIA: Evaluation on the CERV log}
\label{subsec:cerv}

CERV is an event log that describes the process of cervical cancer screening in an Italian screening center, and it has been used in a previous related work \cite{??,deviant-tkde}. The screening program is composed of five phases, organized sequentially: screening planning, invitation management, first level test with pap-test, second level test with colposcopy (only if the first test is positive), and eventually biopsy (if the second test gives a positive response). A number of subjects do not respect the planned protocol: for example, subjects do not show up at the first test, even if they chose the temporal slot themselves. Moreover, a number of subjects prefer to consult other physicians they trust more, if the second or the third test gave a positive response. All these cases provided a great variety of process instances that, from the viewpoint of the manager, are not compliant with the intended protocol, as it commonly happens in socio-technical systems.
%
Hence, the traces have been labeled by a domain expert as belonging either to the positive or the negative set, depending on their compliance to the adopted protocol.


%All models returned by NegDis satisfy the whole positive log and violate all the negative traces. 
In this example, we applied the preferences approach over the “precedence” and “succession” Declare templates. The two first resulting models for, respectively, precedence and succession are as follows:

M1: {    alternateresponse(send positive pap test result, take a colposcopy examination), 
chainprecedence,invito(take a pap test examination), 
exclusivechoice(send pap test sample reject),
precedence(send colposcopy uncertain result, send biopsy sample) }

M2: {     alternateresponse(send positive pap test result, take a colposcopy examination),
chainprecedence,invito(take a pap test examination),
exclusivechoice(send pap test sample, reject),
succession(send colposcopy uncertain result, send biopsy sample) }

These are identical except for the constraints targeted by the preferences, that give a slightly different meaning to the models. In the first model, the precedence constraint implies that a biopsy is executed only if the colposcopy examination has an uncertain result, while the succession constraint in the second model also entails that the biopsy is always executed in those circumstances, giving an additional information.
However, the second model would not exist without the preference expression, because “precedence” subsumes “succession” and the generality criterion would prune it [la sussunzione e il criterio di generalità li assumo già spiegati, vero?].  

A second example on this event log aims to demonstrate the important contribution of a negative preference, that is, a preference for the absence of a certain activity or template in the model. We applied this kind of preference on the “reject” action and we didn’t obtain any model satisfying the desired criterion. However, this result still offers a valuable information, as it guarantees that the “reject” activity is essential to distinguish between the positive and negative traces.



%!TEX root = ./main.tex

\section{Related Work}
\label{sec:related}

\btext{Bologna? Related work di deviant accorciati + i danesi + deviant?}



When process models are loosely-structured, procedural discovery could produce spaghetti-models~\cite{2012-Maggi,deviant-tkde}. In that case, declarative approaches are more suitable for the purpose because they allow to briefly list all the required or prohibited behaviours in the business process. 

Over the last decade, several works focused on declarative process discovery~\cite{2011-Maggi,2012-Maggi,2012-Schunselaar,2017-DiCiccio,2015-DiCiccio}.
In~\cite{2011-Maggi}, Maggi et al. propose to build a set of all possible candidate Declare constraints considering all the activities that appear in the log, translate the constraints in \ac{LTL}, and check them against the whole log until certain levels of recall and specificity are reached. 
Techniques to refine the business model excluding vacuously satisfied constraints are the focus of the subsequent works by Schunselaar et al. \cite{2012-Schunselaar} and Maggi et al. \cite{2012-Maggi}, whereas Di Ciccio et al. \cite{2017-DiCiccio} propose an approach to filter out frequent redundancies and inconsistencies. 
All the cited declarative approaches do not deal with negative examples. Nonetheless, interestingly from our point of view, in \cite{2015-DiCiccio}  the authors define among the metrics to guide the declarative discovery approach, besides support and confidence, also the interest factor for each constraint w.r.t. the log, and the possibility to include in the search space constraints on prohibited behaviours.

Negative examples are instead actively employed in the declarative discovery approaches \cite{2007-Lamma,2009-Chesani,2010-Bellodi,2016-Bellodi,DBLP:conf/bpm/SlaatsDB21}, %, which are all based on the functioning principles of \ac{ICL} algorithm \cite{1995-DaRaedt} and thus intrinsically dependent to the availability of both negative and positive examples. 
eventually relaying for the evaluation on the output of synthetical log generators able to produce both positive and negative process cases \cite{2009-Goedertier,2014-Stocker,2010-Hee,2019-Chesani,2017-Chesani,2020-Loreti}.
The technique by Lamma et al. \cite{2007-Lamma,2007b-Lamma} learns integrity constraints expressed as logical formulas, and translates them into the equivalent DecSerFlow constructs \cite{2006-Aalst}. 
Bellodi et al. \cite{2010-Bellodi}, \cite{2016-Bellodi} employ the same approach and automatically convert the results into Markov Logic formulas---statistical relational learning is used to determine the weight of each formula.
Analogously, Chesani et al. \cite{2009-Chesani} propose to learn a set of SCIFF rules \cite{2008-Alberti} and translate them into ConDec constraints \cite{2006-Pesic}. The approach that we adopt in this work instead, is the one presented in \cite{deviant-tkde}, which directly learns Declare constraints without any intermediate language.
Besides, this adopted approach is grounded on a SAT-based solver analogously to the works \cite{2018-Neider,2019-Camacho,2019-Riener}, where simple \ac{LTL} formulas are generated analysing a set of positive and negative examples.
%\cite{2018-Neider} resort to decision trees to improve the performance while dealing with large input logs; \cite{2019-Camacho} exploit \ac{AFA} to learn \ac{LTL} formulas consistent with positive and negative example sets; and \cite{2019-Riener} propose to split the search space into smaller subproblems using partial \acp{DAG}.

Our notion of negative example is similar to the definitions of syntactical and semantic noise of \cite{2009-Gunther}, because our approach is able to extract both the syntactic information that characterise the positive examples w.r.t. negative, and the relevant semantic difference between traces that has been partially or totally modified at a certain point in time.
%As a minor point, we also might notice that these works provides in output LTL formulas, while we opt for Declare formulas with LTL$_f$ semantics.
%
In this sense, our work is also closely related to deviance mining approaches \cite{2016-Nguyen}, i.e., techniques to extract the relevant details characterising those traces showing a sequence of activities that deviates from the expected behaviour. Wheres some deviance mining approaches \cite{2014-Suriadi,2014-Armas} focus on the differences between models discovered from deviant and non-deviant traces, others \cite{2013-Suriadi,2015-Partington,2013-Bose,2007-Lo,2016-Bernardi} intend it as a sort of sequence classification: to discover patterns of activities that distinguish different types of traces.

%inserting user preferences in the discovery of declarative process models  as a two-class supervised task. In particular we start from a recent work \cite{deviant-arxiv} that introduces the \nd binary miner for the \declare \cite{2009-Aalst} modeling language. We show how to extend it with preference mechanisms able to increase the role of the users in obtaining their preferred models
 
Particularly relevant for our work was the contribution by Slaats et al. \cite{DBLP:conf/bpm/SlaatsDB21}, which proposes a binary classification procedure for process discovery evaluated on a precious set of real-life logs with negative examples from industry.\tododl{vogliamo dire altro di Slaats?}
%https://link.springer.com/chapter/10.1007%2F978-3-030-21290-2_37
%
%https://link.springer.com/chapter/10.1007%2F978-3-642-32885-5_6
%http://ceur-ws.org/Vol-1021/paper_10.pdf
%
%negation of declare constraints:
%https://www.researchgate.net/publication/284570318_Patterns_for_a_Log-Based_Strengthening_of_Declarative_Compliance_Models
%
%process mining ibrido
%https://link.springer.com/article/10.1007/s13740-020-00112-9


%!TEX root = ./main.tex


\section{Conclusions}
\label{sec:conclusions}

In this paper, we address the problem of inserting user preferences in the discovery of declarative process models as a two-class supervised task. In particular, we extend the \nd binary miner for the \declare modeling language with preferences over \declare templates and activities appearing in the model (plus a combination of the two). 
The computation of the preferred models, which take into account the preferences posed by the user, is performed using \asprin, a general framework for computing optimal ASP models with preferences. The provided approach is described by means of motivating examples and an application to the real-life event logs DREYERS and CERV \btext{shows} how to describe - in a discriminative manner - execution traces on the basis of preferred %(e.g., important, ...) 
activities and \declare patterns. Future works will include a wider evaluation, which will also involve end-users. This will enable the assessment of the potential benefits of involving users (through their preferences) in the loop of process discovery.



% \input{metamodel}
%
% \input{discussion}
%
% \input{solutions}
%
% \input{related_works}
%
%
% \input{conclusion}

%
% ---- Bibliography ----
%
% BibTeX users should specify bibliography style 'splncs04'.
% References will then be sorted and formatted in the correct style.
%
 \bibliographystyle{splncs04}
 \bibliography{deviant}
%



\end{document}
