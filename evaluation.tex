








%!TEX root = ./main.tex

\section{Evaluating the discovery}
\label{sec:evaluation}

% \btext{@all? (in particolare chiara DF, FAbrizio, Marco e sergio per pensarla). Pensare alla valutazione. Ho recuperato i due dataset dei danesi (vedi git). Se volete si possono usare questi, pero' uno dei due log in realta' e' un insieme di 215 logs di cui loro hanno fatto la media. Forse io ne sceglierei due o tre (es Dreyers Foundation.xes + uno o due dei 215) e farei la valutazione su questi. Anche perche' piu' che valutazione e' far vedere cosa esce perche' nn mi e' chiaro come valutiamo n termini di metriche l'output.}

In Section \ref{sec:example} we introduced the preference types through simple toy-like examples. The interested reader however might wonder about the usability and efficacy of our approach w.r.t. real-life cases. We explored the applicability of our approach to two real-life event logs, namely DREYERS (492 positive traces and 208 negative ones) and CERV (55 positive traces and 102 negatives traces). In both cases we were able to find ten models satisfying the given preferences in a computation time between 1 and 3 seconds, using a normally-equipped laptop.



\subsection{The DREYERS log}
\label{subsec:dreyers}

The DREYERS log describes the Dreyer Foundation’s processes, regarding their support to legal and architectural projects, and it has been used in \cite{DBLP:conf/ssci/DeboisS15,DBLP:conf/bpm/SlaatsDB21}. Each application that aims to the Foundation's support goes through a pre-screen that can lead to an initial rejection. The remaining applications undergo a review, in which at least one of the reviewers must be a lawyer or an architect, depending on the application type. The review phase is followed by a board meeting, where applications to be supported are selected: in case, the application is funded.

Two sets of log traces are available: process instances that execute properly have been labeled as the positive example set, while process instances that do not adhere to the proper execution path belong to the negative example set. The labeling was already available, and we are not aware of the adopted criteria.

Hence, we plaid a sort of ``investigation game'', and explored the hypothesis that the type of application (architect- or lawyer- type) might affect the process outcome. More generally, one might be interested in knowing if there are any differences in models that explicitly mention one of the two possible types of application.
%, so as to understand the possible variations in the process, depending on the legal or architectural context
%%%To distinguish (traces referring to) the two types of applications, we resorted to focus on two activities peculiar of each application type, namely the “Lawyer Review” and the “Architect Review” activity.
To this end, we specified a preference for models containing the activity \textsf{Lawyer Review}, and then a preference for the \textsf{Architect Review} activity.

In both cases several models satisfying the preferences were found. However, the two sets of models are identical (except for the architect/lawyer activity), showing that the process is probably independent of the application domain, as expected from the process description.
We report an example of a model obtained when specifying a preference for the \textsf{Architect Review} activity:

%l'ordine di restituzione è molto diverso e i primi restituiti sono quelli con l'azione in danese, quindi ne ho scelti due volutamente diversi... è meglio metterli uguali con solo l'azione architect/lawyer diversa?

\begin{align*}
\mathsf{M_1} = \{ \ & \mathsf{ alternateresponse(Undo\ payment, First\ payout)} \\
& \mathsf{ chainprecedence(Fill\ out\ application, Initial\ Rejection)} \\
& \mathsf{ choice(Round\ ends, Change\ phase\ to\ Abort)} \\
& \mathsf{ notchainsuccession(Receive\ final\ report, First\ payout)} \\
& \mathsf{ notchainsuccession(Change\ phase\ to\ Preparation, Approve\ application)} \\
& \mathsf{ notchainsuccession(Change\ phase\ to\ Preparation, Execute\ Pre\ decision)} \\
& \mathsf{ notchainsuccession(Set\ to\ Pre\ approved, Round\ Ends)} \\
& \mathsf{ notsuccession(Architect\ Review, Approval\ on\ to\ the\ board)} \ \} 
\end{align*}

%%%The DREYERS log consists of 492 positive traces and 208 negative ones, 130 positive paths and 29 negative ones. NegDis is able to return models that satisfy the whole positive log and violate all but three negative traces (two paths). 
%%%For the models of the previous example we report the number of traces and paths violated by each constraint:

%%%Model M1:\\
\begin{table}
\label{table:resultsModelM1}
\begin{center}
\begin{scriptsize}
\begin{tabular}{lcc}
%\hline
Constraint & Traces & Paths \\
\hline
alternateresponse(Undo payment, First payout) & 2 & 2\\
%\hline
chainprecedence(Fill out application, Initial Rejection) & 3 & 2\\
%\hline
choice(Round ends, Change phase to Abort) & 195 & 17 \\
%\hline
notchainsuccession(Receive final report, First payout) & 1 & 1 \\
%\hline
notchainsuccession(Change phase to Preparation, Approve application) & 1 & 1\\
%\hline
notchainsuccession(Change phase to Preparation, Execute Pre decision) & 2 & 2 \\
%\hline
notchainsuccession(Set to Pre approved, Round Ends) & 2 & 2\\
%\hline
notsuccession(Architect Review, Approval on to the board) & 1 & 1\\
\hline
Traces not ruled out by the model & 3 & 2 \\
\hline
Total & 208 & 30\\
\hline
\end{tabular}
\end{scriptsize}
\end{center}
\caption{Traces ruled out by model $\mathsf{M_1}$.}
\end{table}
%
Notably, the model is able to discriminate between positive and negative examples except for three negative traces (two paths), that cannot be ruled out without discarding also some positive examples.

We continued our investigation by focusing on the very beginning of the process, and on the specific activity \textsf{Initial Rejection}. Many models were found, among them the following one:

\begin{align*}
\mathsf{M_2} = \{ \ &  \mathsf{ absence2(Initial\ rejection)} \\
& \mathsf{ choice(Round\ Ends, Applicant\ informed)} \\
& \mathsf{ notchainsuccession(Set\ to\ Pre\ approved, Round\ Ends)} \\
& \mathsf{ notchainsuccession(Receive\ final\ report, First\ payout)} \\
& \mathsf{ notchainsuccession(Change\ phase\ to\ Preparation, Approve\ application)} \\
& \mathsf{ notchainsuccession(Change\ phase\ to\ Preparation, Execute\ Pre\ decision)} \\
& \mathsf{ notsuccession(Lawyer\ Review, Change\ phase\ to\ review)} \\
& \mathsf{ response(Undo\ payment, First\ payout) } \ \} 
\end{align*}

Model $\mathsf{M_2}$ highlights the fact that some negative traces can be distinguishable from the positive ones because of the repetition of the \textsf{Initial Rejection}: indeed, some traces reported the execution of the activity twice, thus indicating an attention point for the process manager.

%%%Model M2:\\
%%%
%%%\begin{tabular}{|l|l|l|}
%%%\hline
%%%Constraint & Traces & Paths \\
%%%\hline
%%%absence2(Initial rejection) & 3 & 2\\
%%%\hline
%%%choice(Round ends, Change phase to Abort) & 195 & 17 \\
%%%\hline
%%%notchainsuccession(Receive final report, First payout) & 1 & 1 \\
%%%\hline
%%%notchainsuccession(Change phase to Preparation, Approve application) & 1 & 1\\
%%%\hline
%%%notchainsuccession(Change phase to Preparation, Execute Pre decision) & 2 & 2 \\
%%%\hline
%%%notchainsuccession(Set to Pre approved, Round Ends) & 2 & 2\\
%%%\hline
%%%notsuccession(Lawyer Review, Change phase to review) & 1 & 1\\
%%%\hline
%%%response(Undo payment, First payout) & 2 & 2\\
%%%\hline
%%%\end{tabular}
%%
%%
%%
%%%\btext{DA FINIRE DOPO AVER PARLATO CON GIULIA}


\subsection{Evaluation on the CERV log}
\label{subsec:cerv}

CERV is an event log that describes the process of cervical cancer screening in an Italian screening center, and it has been used in previous related works \cite{2007b-Lamma,deviant-tkde}. The screening program is composed of five phases, organized sequentially: screening planning, invitation management, first level test with pap-test, second level test with colposcopy (only if the first test is positive), and eventually biopsy (if the second test gives a positive response). Several subjects do not respect the planned protocol: e.g., subjects do not show up at the first test, even if they chose the temporal slot themselves. Moreover, a number of subjects prefer to consult physicians they trust more, in case of a positive response. As it commonly happens in socio-technical systems, a great variety of process instances appears in the log, nott all the instances being compliant with the protocol.
%
Hence, the traces have been labeled by a domain expert as belonging either to the positive or the negative set, depending on their compliance to the adopted protocol.

We investigated the log by eliciting a preference over the \textsf{precedence} and \textsf{succession} templates. The first resulting models, respectively, are as follows:

\begin{align*}
\mathsf{M_1} = \{ \ &  \mathsf{ alternateresponse(send\ positive\ pap\ test\ result, take\ a\ colposcopy\ examination)} \\
& \mathsf{ chainprecedence(invite, take\ a\ pap\ test\ examination)} \\
& \mathsf{ exclusivechoice(send\ pap\ test\ sample, reject)} \\
& \mathsf{precedence(send\ colposcopy\ uncertain\ result, send\ biopsy\ sample)} \ \}  \\
\mathsf{M_2} = \{ \ &  \mathsf{ alternateresponse(send\ positive\ pap\ test\ result, take\ a\ colposcopy\ examination)} \\
& \mathsf{ chainprecedence(invite, take\ a\ pap\ test\ examination)} \\
& \mathsf{ exclusivechoice(send\ pap\ test\ sample, reject)} \\
& \mathsf{succession(send\ colposcopy\ uncertain\ result, send\ biopsy\ sample)} \ \}  
\end{align*}

In the first model, the \textsf{precedence} constraint implies that if a biopsy is executed, then the colposcopy examination would have provided an uncertain result before. Surprisingly, the second model is identical to the first one, except for the constraint subjected to our preference. Interestingly, the \textsf{succession} is about the same activities involved by the \textsf{precedence} constraint in the first model. A further observation lies in the logical relation between \textsf{precedence} and \textsf{succession}: a trace that violates the former will always violate the latter (but not the opposite). It is up to the domain expert to prefer a stricter or a more general constraint.
%, knowing that in this case such a choice would not affect the resulting model.

%%%For the models of the previous example we report the number of traces and paths violated by each constraint:
%%%
%%%Model M1:\\
%%%
%%%\begin{tabular}{|l|l|l|}
%%%\hline
%%%Constraint & Traces & Paths \\
%%%\hline
%%%alternateresponse(send positive pap test result, take a colposcopy examination) & 4 & 1\\
%%%\hline
%%%chainprecedence(invite, take a pap test examination) & 3 & 2\\
%%%\hline
%%%exclusivechoice(send pap test sample, reject) & 94 & 2 \\
%%%\hline
%%%precedence(send colposcopy uncertain result, send biopsy sample) & 4 & 3 \\
%%%\hline
%%%\end{tabular}
%%%
%%%Model M2:\\
%%%
%%%\begin{tabular}{|l|l|l|}
%%%\hline
%%%Constraint & Traces & Paths \\
%%%\hline
%%%alternateresponse(send positive pap test result, take a colposcopy examination) & 4 & 1\\
%%%\hline
%%%chainprecedence(invite, take a pap test examination) & 3 & 2 \\
%%%\hline
%%%exclusivechoice(send pap test sample, reject) & 94 & 2 \\
%%%\hline
%%%succession(send colposcopy uncertain result, send biopsy sample) & 4 & 3\\
%%%\hline
%%%\end{tabular}

%%%A second example on this event log aims to demonstrate the important contribution of a negative preference, that is, a preference for the absence of a certain activity or template in the model. We applied this kind of preference on the “reject” action and we didn’t obtain any model satisfying the desired criterion. However, this result still offers a valuable information, as it guarantees that the “reject” activity is essential to distinguish between the positive and negative traces.

%%%In order to evaluate the performances of our approach we also report, for each preference of the previous examples, the time needed to retrieve different numbers of models.\\
%%%\begin{tabular}{|c|c|c|c|c|c|}
%%%\cline{3-6}
%%%\multicolumn{2}{c}{} & \multicolumn{4}{|c|}{Required Time (s)}\\
%%%\hline
%%%\multicolumn{2}{|c}{Log} & \multicolumn{2}{|c}{DREYERS} & \multicolumn{2}{|c|}{CERV}\\
%%%\hline
%%%\multicolumn{2}{|c|}{Preference} & architect activity & lawyer activity & precedence template & succession template\\
%%%\hline
%%%\multirow{5}*{Number of models} & 1 & 0.185 & 0.178 & 0.273 & 0.281\\
%%%\cline{2-6}
%%%& 5 & 0.285 & 0.439 & 0.646 & 0.666\\
%%%\cline{2-6}
%%%& 10 & 0.849 & 1.635 & 1.106 & 3.545\\
%%%\cline{2-6}
%%%& 20 & 2.938 & 3.517 & 3.066 & 18.159\\
%%%\cline{2-6}
%%%& 50 & 14.167 & 22.485 & 76.766 & 130.256\\
%%%\hline
%%%\end{tabular}
%%%
%%%This table shows that the execution time depends both on the log and the specific preference.



